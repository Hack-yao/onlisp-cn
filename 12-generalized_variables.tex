%!TEX encoding = UTF-8 Unicode
% $Id: 12-generalized_variables.tex 18 2014-03-12 22:35:24Z binghe $

\chapter{\gv{}}
\label{chap:generalized_variables}
\index{generalized variables \gv{}}

第~\ref{chap:when_to_use_macros} 章曾提到,宏的长处之一是其变换参数的能力。\verb|setf|\index{setf@\texttt{setf}} 就是
这类宏中的一员。本章将着重分析~\verb|setf| 的内涵,然后以几个宏为例,它们
将建立在~\verb|setf| 的基础之上。

要在~\verb|setf| 上编写正确无误的宏并非易事,其难度让人咋舌。为了介绍这个主题,第一节会先给出一个有点
小问题的简单例子。接下来的小节将解释该宏的错误之处,然后展示如何改正它。第三和第四节会介绍一些基于
~\verb|setf| 的\utility{}的例子,而最后一节则会说明如何定义你自己的~\verb|setf|
\inversion。

\section{概念}
\label{sec:gv:the_concept}

内置宏~\verb|setf| 是~\verb|setq| 的推广形式。\verb|setf| 的第一个参数可以
是个函数调用而非简单的变量:
\begin{lstlisting}
> (setq lst '(a b c))
(A B C)
> (setf (car lst) 480)
480
> lst
(480 B C)
\end{lstlisting}

一般而言,\texttt{(setf $x$ $y$)} 可以理解成~``务必让~$x$ 的求值结果为~$y$''。
作为一个宏,\verb|setf| 得以深入到参数内部,弄清需要做哪些工作,才能满足这个要求。
如果第一个参数~(在宏展开以后) 是个符号,那么~\verb|setf| 就只会展开成
~\verb|setq|。但如果第一个参数是个查询语句,那么~\verb|setf| 则会展开到对应的断言上。
由于第二个参数是常量,所以前面的例子可以展开成:
\begin{lstlisting}
(progn (rplaca lst 480) 480)
\end{lstlisting}
\index{rplaca@\texttt{rplaca}}
这种从查询到断言的变换被称为\emph{\inversion}。Common Lisp 中所有最常用的访问函数都有
预定义的\inversion,包括~\verb|car|、\verb|cdr|、\verb|nth|、
\verb|aref|、\verb|get|、\verb|gethash|,以及那些由~\verb|defstruct| 创建的访问函数。(
完整的名单见~\textsc{cltl}2 的第~125 页。)

能充当~\verb|setf| 第一个参数的表达式被称为\emph{\gv{}}。\gv{}已经成为了一种强有力的
抽象机制。宏调用和\gv{}的相似之处在于:一个宏调用,只要能展开成可逆引用,那么其本身就一定是可逆的\index{macros 宏!calls invertible 可逆调用}。

当我们也加入这个行列,基于~\verb|setf| 编写自己的宏时,这种组合可以产生显而易见更清爽的程序。
我们可以在~\verb|setf| 上面定义的宏有很多,其中一个是~\verb|toggle|\footnote{这个定义是
错误的,下一节将给出解释。},
\begin{lstlisting}
(defmacro toggle (obj)
  `(setf ,obj (not ,obj)))
\end{lstlisting}
它可以反转一个\gv{}的值:
\begin{lstlisting}
> (let ((lst '(a b c)))
    (toggle (car lst))
    lst)
(NIL B C)
\end{lstlisting}

现在考虑下面的应用。假设有个人,他可能是个肥皂剧作者、精力充沛的好事者,或是居委会大妈,
想要维护一个数据库。其中记录着小镇上所有居民之间的种种恩怨情仇。在数据库里的表里,
其中有一张便是用来保存朋友关系的:
\begin{lstlisting}
(defvar *friends* (make-hash-table))
\end{lstlisting}
这个哈希表的表项本身也是哈希表,其中,潜在的朋友被映射到~\texttt{t} 或者~\texttt{nil}:
\begin{lstlisting}
(setf (gethash 'mary *friends*) (make-hash-table))
\end{lstlisting}
为了使~John 成为~Mary 的朋友,我们可以说:
\begin{lstlisting}
(setf (gethash 'john (gethash 'mary *friends*)) t)
\end{lstlisting}

这个镇被分为两派\index{factions 派系}。正如帮派的传统,每个人都声称~``凡人非友即敌'',所以镇上
所有人都被迫加入一方或者另一方。这样当某人转变立场时,他所有的朋友都变成敌人,而所有的敌人则
变成朋友。

如果只用内置的操作符来切换~\verb|x| 和~\verb|y| 的敌友关系,我们必须这样说:
\begin{lstlisting}
(setf (gethash x (gethash y *friends*))
      (not (gethash x (gethash y *friends*))))
\end{lstlisting}
尽管去掉~\verb|setf| 后要简单许多,这个表达式还是相当复杂。倘若我们为数据库定义了
一个访问宏\index{macros 宏!access},如下:
\begin{lstlisting}
(defmacro friend-of (p q)
  `(gethash ,p (gethash ,q *friends*)))
\end{lstlisting}
那么在这个宏和~\verb|toggle| 的协助下,我们就得以更方便地修改数据库的数据。
前面那个更新数据库的语句可以简化成:
\begin{lstlisting}
(toggle (friend-of x y))
\end{lstlisting}

\gv{}就像是美味的健康食品。它们能让你的程序良好地模块化\index{modularity 模块化},同时变得更为优雅。
如果你给出宏或者可逆函数,用来访问你的数据结构,那么其他模块就可以使用~\verb|setf| 来修改你的数据结构而无需
了解其内部细节。

\section{多重求值问题}
\label{sec:the_multiple_evaluation_problem}

上一节曾警告说,我们最初的~\verb|toggle| 定义是不正确的:
\begin{lstlisting}[escapechar=\@]
(defmacro toggle (obj)                  @\hfill@; wrong
  `(setf ,obj (not ,obj)))
\end{lstlisting}
它会碰到第~\ref{sec:number_of_evaluations} 节里提到的多重求值问题。如果它的
参数有副作用,那麻烦就来了。比如说,若~\verb|lst| 是一个对象列表,我们这样写:
\begin{lstlisting}
(toggle (nth (incf i) lst))
\end{lstlisting}
\index{incf@\texttt{incf}}
并期待它能反转第~\verb|(i+1)| 个元素。事与愿违,如果使用~\verb|toggle| 现在的定义,
这个调用将展开成:
\begin{lstlisting}
(setf (nth (incf i) lst)
      (not (nth (incf i) lst)))
\end{lstlisting}
这会使~\verb|i| 递增两次,并且将第~\verb|(i+1)| 个元素设置成第~\verb|(i+2)|
个元素的反。所以在本例中
\begin{lstlisting}
> (let ((lst '(t nil t))
        (i -1))
    (toggle (nth (incf i) lst))
    lst)
(T NIL T)
\end{lstlisting}
调用~\verb|toggle| 毫无效果。

仅仅把作为~\verb|toggle| 参数给出的表达式插入到~\verb|setf| 的第一个参数的位置上
还不够。我们必须深入到表达式内部,看看它到底做了什么:如果它含有~subform,而且这些~subform 有副作用的话,
我们就需要把它们分开,并单独求值。一般而言,这件事情并不那么简单。

为了让问题容易些,Common Lisp 提供了一个宏,它可以帮助我们自动定义一些基于~\verb|setf| 的宏,不过适用范围有限。宏的
名字叫~\verb|define-modify-macro|\index{define-modify-macro@\texttt{define-modify-macro}},它接受三个参数:被定义宏的宏名,它的附加参数(出现在\gv{}之后),以及一个函数名,这个函数将为\gv{}产生新值。
\footnote{一般意义上的函数名:\texttt{1+} 或者~\texttt{(lambda (x) (+ x 1))} 都可以。}\footnote{译者注:现行~Common Lisp 标准~(\textsc{CLHS}) 事实上要求~\texttt{define-modify-macro} 和~\texttt{define-compiler-macro} 的第三个参数的类型必须是符号。}

使用~\verb|define-modify-macro|,我们可以像下面这样定义~\verb|toggle|:
\begin{lstlisting}
(define-modify-macro toggle () not)
\end{lstlisting}
具体说,就是~``若要求值形如~\texttt{(toggle \emph{place})} 的表达式,
应该先找到~\emph{place} 指定的位置,并且,如果保存在那里的值是~\emph{val},将其替换成
~\texttt{(not \emph{val})} 的值''。下面把这个新宏用在原来的例子里:
\begin{lstlisting}
> (let ((lst '(t nil t))
        (i -1))
    (toggle (nth (incf i) lst))
    lst)
(NIL NIL T)
\end{lstlisting}

虽然这个版本正确无误地给出了结果,但它本可以写得更通用些。由于~\verb|setf| 和~\verb|setq| 两者
对其参数数量都没有限制,\verb|toggle| 也应如此。我们可以通过在修改宏~(modify-macro)
的基础上定义另一个宏,来赋予它这种能力,如图~\ref{fig:macros_which_operate_on_generalized_variables} 所示。

\begin{figure}
\begin{lstlisting}
(defmacro allf (val &rest args)
  (with-gensyms (gval)
    `(let ((,gval ,val))
       (setf ,@(mapcan #'(lambda (a) (list a gval))
                       args)))))

(defmacro nilf (&rest args) `(allf nil ,@args))

(defmacro tf (&rest args) `(allf t ,@args))

(defmacro toggle (&rest args)
  `(progn
     ,@(mapcar #'(lambda (a) `(toggle2 ,a))
               args)))

(define-modify-macro toggle2 () not)
\end{lstlisting}
\index{allf@\texttt{allf}}
\index{nilf@\texttt{nilf}}
\index{tf@\texttt{tf}}
\index{toggle@\texttt{toggle}}
  \caption{操作在\gv{}上的宏}
  \label{fig:macros_which_operate_on_generalized_variables}
\end{figure}

\section{新的\utility{}}
\label{sec:new_utilities}

本节将给出一些新的\utility{}为例,我们用它们对\gv{}进行操作。这些\utility{}必须是宏,
以便将参数原封不动地传给~\verb|setf|。

图~\ref{fig:macros_which_operate_on_generalized_variables} 中有四个基于
~\verb|setf| 的新宏。第一个是~\verb|allf|,它被用来将同一值赋给多个\gv{}。
\verb|nilf| 和~\verb|tf| 就是基于它实现的,它们分别将参数设置为~\verb|nil|
和~\verb|t|。虽然这些宏很简单,但是方便实用。

和~\verb|setq| 一样,\verb|setf| 也可以接受多个参数\pozhehao{}即交替出现的变量和对应的值:
\begin{lstlisting}
(setf x 1 y 2)
\end{lstlisting}
这些新的\utility{}同样有这个能力,而且只用传原来一半的参数就可以了。如果你想要把多个变量初始化为
~\verb|nil|,那么可以不再使用
\begin{lstlisting}
(setf x nil y nil z nil)
\end{lstlisting}
而改成说
\begin{lstlisting}
(nilf x y z)
\end{lstlisting}
就行了。最后一个宏是前一节曾介绍过的~\verb|toggle|:它和~\verb|nilf| 差不多,但
给每个参数设置的是真值的反。

这四个宏说明了关于赋值操作符的一个要点\index{assignment 赋值!macros for}。就算我们只需要对普通变量使用一个操作符,
而把这个操作符号展开成~\verb|setf| 而非~\verb|setq|,这样做,有百利而无一害。如果第一个参数是符号,
\verb|setf| 将直接展开到~\verb|setq|。由于不费吹灰之力,就能拥有
~\verb|setf| 的一般性,所以很少有必要在展开式里使用~\verb|setq|\index{setq@\texttt{setq}!now redundant}。

\begin{figure}
\begin{lstlisting}
(define-modify-macro concf (obj) nconc)

(defun conc1f/function (place obj)
  (nconc place (list obj)))

(define-modify-macro conc1f (obj) conc1f/function)

(defun concnew/function (place obj &rest args)
  (unless (apply #'member obj place args)
    (nconc place (list obj))))

(define-modify-macro concnew (obj &rest args)
  concnew/function)
\end{lstlisting}
  \caption{\gv{}上的列表操作}
  \label{fig:list_operations_on_generalized_variables}
  \index{concf@\texttt{concf}}
  \index{concnew@\texttt{conc1f}}
  \index{concnew@\texttt{concnew}}
\end{figure}

图~\ref{fig:list_operations_on_generalized_variables}\footnote{
译者注:这里根据现行~Common Lisp 标准对源代码加以修改,我们额外定义了两个辅助函数以确保
~\texttt{define-modify-macro} 的第三个参数只能是符号。} 包含三个破坏性修改列表结尾\index{lists!operating on end of}的
宏。第~\ref{sec:functional_design} 节提到依赖
\begin{lstlisting}
(nconc x y)
\end{lstlisting}
的副作用是不可靠的,并且必须改成\footnote{译者注:当作为~\texttt{nconc} 第一个参数的
变量为空列表,也就是~\texttt{nil} 时,该变量在~\texttt{nconc} 执行之后将仍是
~\texttt{nil},而不是整个~\texttt{nconc} 表达式的那个相当于其第二个参数的值。}
\begin{lstlisting}
(setq x (nconc x y))
\end{lstlisting}
这一习惯用法被嵌入在~\verb|concf| 中了。更特殊的~\verb|conc1f| 和
~\verb|concnew| 就像是用于列表另一端的~\verb|push| 和~\verb|pushnew|:
\verb|conc1f| 在列表结尾追加一个元素,而~\verb|concnew| 的功能相同,但只有当
这个元素不在列表中时才会动作。

第~\ref{sec:defining_functions} 节曾提到,函数的名字既可以是符号,也可以是~\lexpr。
因此,把整个~\lexpr 作为第三个参数传给~\verb|define-modify-macro| 也是可行的,正如
~\verb|conc1f| 的定义。\footnote{译者注:正如前面两个脚注里提到的那样,Common
Lisp 标准并没有定义~\texttt{define-modify-macro} 的第三个参数可以是符号之外的其他东西
,尽管~\lexpr 出现在一个函数调用形式的函数位置上确实是合法的。原书作者试图通过类比来说明
~\lexpr 用在~\texttt{define-modify-macro} 中的合法性,这是不恰当的,请读者注意。}
如果用~\pageref{fig:small_functions_which_operate_on_lists} 页上的~\verb|conc1| 的话
,这个宏也可以写成:
\begin{lstlisting}
(define-modify-macro conc1f (obj) conc1)
\end{lstlisting}

在一种情况下,图~\ref{fig:list_operations_on_generalized_variables} 中的宏应该限制使用。
如果你正准备通过在结尾处追加元素的方式来构造列表,那么最好用~\verb|push|,最后
再~\verb|nreverse| 这个列表。在列表的开头处理数据比在结尾要方便些,因为
在结尾处处理数据的话,你首先得到那里。Common Lisp 有许多用于前者的操作符,而适用于后者的操作符则屈指可数,
这很可能是为了鼓励程序员设计更高效率的程序。

\section{更复杂的\utility{}}
\label{sec:more_complex_utilities}

并非所有基于~\verb|setf| 的宏都可以用~\verb|define-modify-macro| 定义。
比如说,假设我们想要定义一个宏~\verb|_f|,让它破坏性把函数应用于一个\gv{}。内置宏
~\verb|incf| 就相当于使用了~\verb|+| 的~\verb|setf| 的缩写。把
\begin{lstlisting}
(setf x (+ x y))
\end{lstlisting}
取而代之,我们只需说
\begin{lstlisting}
(incf x y)
\end{lstlisting}
新的宏~\verb|_f| 就是上述思路的推广:\verb|incf| 能展开成对~\verb|+| 的
调用,而~\verb|_f| 则会展开成对由第一个参数给出操作符的调用。例如,在
第~\pageref{fig:original_move_and_scale} 页~\verb|scale-objs| 的定义里,我们
必须这样写
\begin{lstlisting}
(setf (obj-dx o) (* (obj-dx o) factor))
\end{lstlisting}
改用~\verb|_f| 的话,将变成
\begin{lstlisting}
(_f * (obj-dx o) factor)
\end{lstlisting}
\verb|_f| 可能会被错写成:
\begin{lstlisting}[escapechar=\@]
(defmacro _f (op place &rest args)       @\hfill@; wrong
  `(setf ,place (,op ,place ,@args)))
\end{lstlisting}
不幸的是,我们无法用~\verb|define-modify-macro| 正确无误地定义~\verb|_f|,因为
应用到\gv{}上的操作符是由参数给定的。

这类更复杂的宏必须由手工编写。为了让这种宏的编写方便些,Common Lisp 提供了函数
~\verb|get-setf-expansion|\index{get-setf-expansion@\texttt{get-setf-expansion}}
\footnote{译者注:原书中给出的函数实际上
  是~\texttt{get-setf-method}\index{get-setf-method@\texttt{get-setf-method}},
  但这个函数已经不在现行~Common Lisp 标准中了,参见~\href{www.lisp.org/HyperSpec/Issues/iss308_w.html}{X3J13 Issue 308:
  \texttt{SETF-METHOD-VS-SETF-METHOD}}。取代它的是
  ~\texttt{get-setf-expansion},这个函数接受两个参数,\emph{place} 以及
  可选的~\emph{environment} 环境参数。本书后面对于所有采
  用~\texttt{get-setf-method} 的地方一律直接改
  用~\texttt{get-setf-expansion},不再另行说明。},它接受一个\gv{}并返回所
有用于获取和设置其值的必要信息。通过为下面表达式手工生成展开
式,我们将了解如何使用这些信息:
\begin{lstlisting}
(incf (aref a (incf i)))
\end{lstlisting}

当我们对\gv{}调用~\verb|get-setf-expansion| 时,可以得到五个值用作宏展开式
的原材料:
\begin{lstlisting}
> (get-setf-expansion '(aref a (incf i)))
(#:G4 #:G5)
(A (INCF I))
(#:G6)
(SYSTEM:SET-AREF #:G6 #:G4 #:G5)
(AREF #:G4 #:G5)
\end{lstlisting}
最开始的两个值分别是临时变量列表,以及应该给它们赋的值。因此,我们可以这样开始展开式:
\begin{lstlisting}
(let* ((#:g4 a)
       (#:g5 (incf i)))
  ...)
\end{lstlisting}
这些绑定应该在~\verb|let*|\index{let*@\texttt{let*}} 里创建。因为一般来说,这些值~form 可能会引用到前面的变量。
第三\footnote{第三个值当前总是一个单元素列表。它被返回成一个列表来提供(目前为止
还不可能)
在\gv{}中保存多值的可能性。}和第五个值是另一个临时变量和将返回\gv{}初值的~form。
由于我们想要在这个值上加~1,所以把后者包在对~\verb|1+| 的调用里:
\begin{lstlisting}
(let* ((#:g4 a)
       (#:g5 (incf i))
       (#:g6 (1+ (aref #:g4 #:g5))))
  ...)
\end{lstlisting}
最后,\texttt{get-setf-expansion} 返回的第四个值是一个赋值的表达式,该赋值必须在新绑定环境下进行:
\begin{lstlisting}
(let* ((#:g4 a)
       (#:g5 (incf i))
       (#:g6 (1+ (aref #:g4 #:g5))))
  (system:set-aref #:g6 #:g4 #:g5))
\end{lstlisting}
不过,这个~form 多半会引用一些内部函数\index{functions 函数!internal 内部},而这些内部函数不属于~Common Lisp 标准。
通常~\verb|setf| 掩盖了这些函数的存在,但它们必须存在于某处。因为关于它们的所有东西都依赖于具体的实现,所以
注重可移植性的代码应该使用由~\verb|get-setf-expansion| 返回的这些~form,而不是直接引用诸如
~\verb|system:set-aref| 这样的函数。

现在为实现~\verb|_f| 而编写的宏,所要完成的工作,几乎和我们刚才手工展开~\verb|incf| 时做的事情完全一样。
唯一的区别就是,不再把~\verb|let*| 里的最后一个~form 包装在~\verb|1+|
调用里,而是将它包装在来自~\verb|_f| 参数的一个表达式里。图
~\ref{fig:more_complex_macros_on_setf} 给出了~\verb|_f| 的定义。

\begin{figure}
\begin{lstlisting}
(defmacro _f (op place &rest args)
  (multiple-value-bind (vars forms var set access)
                       (get-setf-expansion place)
    `(let* (,@(mapcar #'list vars forms)
            (,(car var) (,op ,access ,@args)))
       ,set)))

(defmethod pull (obj place &rest args)
  (multiple-value-bind (vars forms var set access)
                       (get-setf-expansion place)
    (let ((g (gensym)))
      `(let* ((,g ,obj)
              ,@(mapcar #'list vars forms)
              (,(car var) (delete ,g ,access ,@args)))
         ,set))))

(defmacro pull-if (test place &rest args)
  (multiple-value-bind (vars forms var set access)
                       (get-setf-expansion place)
    (let ((g (gensym)))
      `(let* ((,g ,test)
              ,@(mapcar #'list vars forms)
              (,(car var) (delete-if ,g ,access ,@args)))
         ,set))))

(defmacro popn (n place)
  (multiple-value-bind (vars forms var set access)
                       (get-setf-expansion place)
    (with-gensyms (gn glst)
      `(let* ((,gn ,n)
              ,@(mapcar #'list vars forms)
              (,glst ,access)
              (,(car var) (nthcdr ,gn ,glst)))
         (prog1 (subseq ,glst 0 ,gn)
                ,set)))))
\end{lstlisting}
  \caption{\texttt{setf} 上更复杂的宏}
  \label{fig:more_complex_macros_on_setf}
  \index{f@\texttt{\_f}}
  \index{pull@\texttt{pull}}
  \index{pull-if@\texttt{pull-if}}
  \index{popn@\texttt{popn}}
  \index{incf@\texttt{incf}!generalization of 的泛化}
\end{figure}

这是个很有用的\utility{}。举个例子,现在在它的帮助下,我们就可以轻易地将任意有名函数替换成
其记忆化(第~\ref{sec:memoizing} 节)\index{memoizing 记忆化} 的等价函数。\footnote{
然而,内置函数是个例外,它们不应该以这种方式被记忆化。Common Lisp 禁止重定义内置函数。\index{functions 函数!redefining built-in}}
要对~\verb|foo| 进行记忆化的处理,可以用:
\begin{lstlisting}
(_f memoize (symbol-function 'foo))
\end{lstlisting}

使用~\verb|_f|,也有助于简化其他基于~\verb|setf| 的宏的定义。例如,我们现在可以把
~\verb|conc1f|(图~\ref{fig:list_operations_on_generalized_variables})
定义成:
\begin{lstlisting}
(defmacro conc1f (lst obj)
  `(_f nconc ,lst (list ,obj)))
\end{lstlisting}

图~\ref{fig:more_complex_macros_on_setf} 中还有其他一些有用的宏,它们同样基于~\verb|setf|。
下一个是~\verb|pull|,它是内置的~\verb|pushnew| 的逆操作。这对操作符,就像是给~\verb|push|
和~\verb|pop| 赋予了一定的鉴别能力。如果给定的新元素不是列表的成员,\verb|pushnew|\index{pushnew@\texttt{pushnew}} 就把它加入到这个列表里面,而~\verb|pull| 则是破坏性地从列表里删除给定的元素。\verb|pull|
定义中的~\verb|&rest|\index{rest@\texttt{\&rest} parameters!in utilities} 参数使~\verb|pull| 可以接受和~\verb|delete| 相同的关键字参数:
\begin{lstlisting}
> (setq x '(1 2 (a b) 3))
(1 2 (A B) 3)
> (pull 2 x)
(1 (A B) 3)
> (pull '(a b) x :test #'equal)
(1 3)
> x
(1 3)
\end{lstlisting}
你几乎可以把这个宏当成这样定义的:
\begin{lstlisting}[escapechar=\@]
(defmacro pull (obj seq &rest args)      @\hfill@; wrong
  `(setf ,seq (delete ,obj ,seq ,@args)))
\end{lstlisting}
不过,如果它真的这样定义,它将同时碰到求值顺序和求值次数方面的问题。我们也可以把
~\verb|pull| 定义成简单的修改宏:
\begin{lstlisting}
(define-modify-macro pull (obj &rest args)
  (lambda (seq obj &rest args)
    (apply #'delete obj seq args)))
\end{lstlisting}
但由于修改宏必须将\gv{}作为第一个参数,所以我们只得以相反的次序给出前两个参数,这样显得
有些不自然。

更通用的~\verb|pull-if| 接受一个初始的函数参数,并且会展开成
~\verb|delete-if| 而非~\verb|delete|:
\begin{lstlisting}
> (let ((lst '(1 2 3 4 5 6)))
    (pull-if #'oddp lst)
    lst)
(2 4 6)
\end{lstlisting}

这两个宏说明了另一个有普遍意义的要点。如果下层函数接受可选参数,建立在其上的宏也应该这样做。
\verb|pull| 和~\verb|pull-if| 都把可选参数传给了它们的~\verb|delete|。

图~\ref{fig:more_complex_macros_on_setf} 中最后一个
宏是~\verb|popn|,它是~\verb|pop| 的推广形式。其功能不再是仅仅从列表里弹
出一个元素,而是能弹出并返回任意长度的子序列:
\begin{lstlisting}
> (setq x '(a b c d e f))
(A B C D E F)
> (popn 3 x)
(A B C)
> x
(D E F)
\end{lstlisting}

\begin{figure}
\begin{lstlisting}
(defmacro sortf (op &rest places)
  (let* ((meths (mapcar #'(lambda (p)
                            (multiple-value-list
                              (get-setf-expansion p)))
                        places))
         (temps (apply #'append (mapcar #'third meths))))
    `(let* ,(mapcar #'list
                    (mapcan #'(lambda (m)
                                (append (first m)
                                        (third m)))
                            meths)
                    (mapcan #'(lambda (m)
                                (append (second m)
                                        (list (fifth m))))
                            meths))
       ,@(mapcon #'(lambda (rest)
                     (mapcar
                       #'(lambda (arg)
                           `(unless (,op ,(car rest) ,arg)
                              (rotated ,(car rest) ,arg)))
                       (cdr rest)))
                 temps)
       ,@(mapcar #'fourth meths))))
\end{lstlisting}
  \caption{一个排序其参数的宏}
  \label{fig:a_macro_which_sorts_its_arguments}
  \index{sortf@\texttt{sortf}}
  \index{sorting!of arguments}
\end{figure}

图~\ref{fig:a_macro_which_sorts_its_arguments} 中的宏能对它
的参数排序。如果~\verb|x| 和~\verb|y| 是变量,而且我们想要确
保~\verb|x| 的值不是两个值中较小的那个,那么我们可以写:
\begin{lstlisting}
(if (> y x) (rotatef x y))
\end{lstlisting}
但如果我们想对三个或者数量更多的变量做这个操作,所需的代码量就会迅速膨胀。与其手工编写这样的代码,不妨让~\verb|sortf| 来为我们代劳。这个宏接受一个比较操作符,还有任意数量的\gv{},然后不断交换它们
的值,直到这些\gv{}的顺序符合操作符的要求。在最简单的情形,参数可以是普通变量:
\begin{lstlisting}
> (setq x 1 y 2 z 3)
3
> (sortf > x y z)
3
> (list x y z)
(3 2 1)
\end{lstlisting}
一般情况下,它们可以是任何可逆的表达式。假设~\verb|cake| 是一个可逆函
数,它能返回某人的蛋糕,而~\verb|bigger| 是个针对蛋糕的比较函数。如
果我们想要推行一个规定,要求~\verb|moe| 的~\verb|cake| 不得小
于~\verb|larry| 的~\verb|cake|,而后者的~\verb|cake| 也不得小
于~\verb|curly| 的,我们写成:
\begin{lstlisting}
(sortf bigger (cake 'moe) (cake 'larry) (cake 'curly))
\end{lstlisting}

\verb|sortf| 的定义的大致结构和~\verb|_f| 差不多。它以一
个~\verb|let*| 开始,在这个~\verb|let*| 表达式中,由~\verb|get-setf-expansion| 返回的临时变量被绑
定到~\gv{} 的初始值上。\verb|sortf| 的核心是中间
的~\verb|mapcon|\index{mapcon@\texttt{mapcon}} 表达式,该表达式生成的代
码将被用来对这些临时变量进行排序。宏的这部分生成的代码量会随着参数个数以指数
速度增长。在排序之后,\gv{}会被用那些由~\verb|get-setf-expansion|返回的~form 
重新赋值。这里使用的算法是~$O(n^2)$ 的冒泡排序,但如果调用的时候参数非常多的话,
这个宏就不适用了。

\begin{figure}
\begin{lstlisting}
(sortf > x (aref ar (incf i)) (car lst))
\end{lstlisting}
展开~(在某个可能的实现里) 成:
\begin{lstlisting}
(let* ((#:g1 x)
       (#:g4 ar)
       (#:g3 (incf i))
       (#:g2 (aref #:g4 #:g3))
       (#:g6 lst)
       (#:g5 (car #:g6)))
  (unless (> #:g1 #:g2)
    (rotatef #:g1 #:g2))
  (unless (> #:g1 #:g5)
    (rotatef #:g1 #:g5))
  (unless (> #:g2 #:g5)
    (rotatef #:g2 #:g5))
  (setq x #:g1)
  (system:set-aref #:g2 #:g4 #:g3)
  (system:set-car #:g6 #:g5))
\end{lstlisting}
  \caption{一个~\texttt{sortf} 调用的展开式}
  \label{fig:expansion_of_a_call_to_sortf}
\end{figure}

图~\ref{fig:expansion_of_a_call_to_sortf} 给出的是对~\verb|sortf| 调用的
展开式。在最前面的~\verb|let*| 中,参数和它们的~subform 按照从左到右的顺
序小心地求值。之后出现的三个表达式分别比较几个临时变量的值,有可能还会
交换它们:先是比较第一个和第二个,接着是第一个和第三个,然后第二个和第
三个。最后\gv{}从左到右被重新赋值。尽管很少需要注意这个问题,但还是提一
下:通常,宏参数应该按从左到右的顺序进行赋值\index{assignment 赋值!order
  of 的顺序} ,这和它们求值的顺序是一致的。

有些操作符,如~\verb|_f| 和~\verb|sortf|,它们与接受函数型参数的函数之间确实有相似之处\index{functions 函数!as arguments}。
不过也应该认识到它们是完全不同的东西。类似~\verb|find-if| 的函数接受一个函数并调用它;
而类似~\verb|_f| 的宏接受的则是一个\emph{名字},这些宏会让它成为一个表达式的~car。
让~\verb|_f| 和~\verb|sortf| 都接受函数型参数也不无可能。例如,\verb|_f| 可以
这样实现:
\begin{lstlisting}
(defmacro _f (op place &rest args)
  (let ((g (gensym)))
    (multiple-value-bind (vars forms var set access)
                         (get-setf-expansion place)
      `(let* ((,g ,op)
              ,@(mapcar #'list vars forms)
              (,(car var) (funcall ,g ,access ,@args)))
         ,set))))
\end{lstlisting}
然后调用~\verb|(_f #'+ x 1)|。但是~\verb|_f| 原来的版本不但拥有这个版本
的所有功能,而且由于它处理的是名字,所以它还可以接受宏或者~special form 的名字。就像
~\verb|+| 那样,比如说,你还可以调用~\verb|nif|
(\pageref{fig:macros_for_conditional_evaluation} 页):
\begin{lstlisting}
> (let ((x 2))
    (_f nif x 'p 'z 'n)
    x)
P
\end{lstlisting}

\section{定义\inversion}
\label{sec:defining_inversions}
\index{inversions!defining}

\ref{sec:gv:the_concept} 节说明了一个道理:如果一个宏调用能展开成可逆引用,那么它本身应该也是可逆的。
不过,你也用不着只是为了可逆,就把操作符定义成宏。通过使用~\verb|defsetf|\index{defsetf@\texttt{defsetf}},你可以告诉~Lisp
如何对任意的函数或宏调用求逆。

使用这个宏的方法有两种。在最简单的情况下,它的参数是两个符号:
\begin{lstlisting}
(defsetf symbol-value set)
\end{lstlisting}
\index{set@\texttt{set}}
\index{symbol-value@\texttt{symbol-value}}
如果用更复杂的方法,那么~\verb|defsetf| 的调用和~\verb|defmacro| 调用会有几分相似,
它另外带有一个参数用于更新值~form。例如,下式可以为~\verb|car| 定义一种可能的\inversion:
\begin{lstlisting}
(defsetf car (lst) (new-car)
  `(progn (rplaca ,lst ,new-car)
          ,new-car))
\end{lstlisting}
\verb|defmacro| 和~\verb|defsetf| 之间有一个重要的区别:后者会自动为其参数创建
生成符号~(gensym)。通过上面给出的定义,\verb|(setf (car x) y)| 将展开成:
\begin{lstlisting}
(let* ((#:g2 x)
       (#:g1 y))
  (progn (rplaca #:g2 #:g1)
         #:g1))
\end{lstlisting}
这样,我们写~\verb|defsetf| 展开器时就没有后顾之忧,
不用担心诸如变量捕捉,或者求值的次数和顺序之类的问题了。

在~\textsc{cltl}2 的~Common Lisp 中,也可以直接用~\verb|defun| 定义
~\verb|setf| 的逆\index{defun@\texttt{defun}!defining inversions with}。因而前面的示例也可以写成:
\begin{lstlisting}
(defun (setf car) (new-car lst)
  (rplaca lst new-car)
  new-car)
\end{lstlisting}
新的值应该作为这个函数的第一个参数。同样按照习惯,也应该把这个值作为函数的返回值。

目前为止的示例都认为,\gv{}应该指向数据结构中的某个位置。\index{generalized variables 广义变量!meaning of}不法之徒把人质带进地牢,而见义勇为之士则让她重见天日;
他们移动的路径相同,但方向相反。所以,如果人们觉得~\verb|setf| 的工作方式也
只能是这样,那不足为奇,因为所有预定义的逆看上去都是如此;确实,习惯上,将被求逆的参数
也常会使用~\verb|place| 作为其参数名。

从理论上说,\verb|setf| 可以更一般化:access form 和它的逆的操作对象甚至可以不是同种数据结构。
假设在某个应用里,我们想要把数据库的更新缓存起来\index{databases 数据库!caching updates to 缓存更新}。
这可能是迫不得已的,举例来说,倘若每次修改数据,都即时完成真正的更新操作,就有可能会降低效率,或者,如果要求所有的更新都必须在提交之前验证一致性,那就必须引入缓存的机制。

\begin{figure}
\begin{lstlisting}
(defvar *cache* (make-hash-table))

(defun retrieve (key)
  (multiple-value-bind (x y) (gethash key *cache*)
    (if y
        (values x y)
        (cdr (assoc key *world*)))))

(defsetf retrieve (key) (val)
  `(setf (gethash ,key *cache*) ,val))
\end{lstlisting}
  \caption{一个非对称的逆}
  \label{fig:an_asymmetric_inversion}
  \index{inversions!asymmetric}
\end{figure}

假设~\texttt{*world*} 是实际的数据库。为简单起见,我们将它做成一个元素为~\texttt{
(\emph{key} . \emph{val})} 形式的关联表~(assoc--list)。图
~\ref{fig:an_asymmetric_inversion} 显示了一个称为~\texttt{retrieve} 的查询函数。
如果~\texttt{*world*} 是
\begin{lstlisting}
((a . 2) (b . 16) (c . 50) (d . 20) (f . 12))
\end{lstlisting}
那么
\begin{lstlisting}
> (retrieve 'c)
50
\end{lstlisting}
和~\verb|car| 的调用不同,\verb|retrieve| 调用并不指向一个数据结构中的特定
位置。返回值可能来自两个位置里的一个。而~\verb|retrieve| 的逆,同样定义在
图~\ref{fig:an_asymmetric_inversion} 中,仅指向它们中的一个:
\begin{lstlisting}
> (setf (retrieve 'n) 77)
77
> (retrieve 'n)
77
T
\end{lstlisting}
该查询返回第二个值~\verb|t|,以表明在缓存中找到了答案。

就像宏一样,\gv{} 是一种威力非凡的抽象机制。这里肯定还有更多的东西有待发掘。当然,有的用户很可能已经发现了
一些使用\gv{}的方法,使用这些方法能得到更优雅和强大的程序。但也不排除以全新的方式使用
~\verb|setf| 逆的可能性,或者发现其它类似的有用的变换技术\note{180}。

%%% Local Variables: 
%%% coding: utf-8
%%% mode: latex
%%% TeX-master: "onlisp-cn"
%%% End: 
