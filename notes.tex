%!TEX encoding = UTF-8 Unicode
% $Id: notes.tex 5 2009-08-07 08:40:43Z binghe $

\chapter{附注}
\label{chap:notes}

本节同时也作为参考文献。所有列在这里的书籍和论文都值得一读。

\begin{notes}
\notei{v}
  Foderaro, John K. Introduction to the Special Lisp
  Section. \emph{CACM} 34, 9 (September 1991), p. 27.

\notei{viii}
  最终的~Prolog 实现有~94 行代码。它使用了来自前面章节的~90 行的\utility。
  \textsc{atn} 编译器增加了~33 行,这样总共是~217 行。由于~Lisp 下的一行
  代码没有标准的表示法,这样当使用行数来衡量一个~Lisp 程序的规模时就会有
  很多偏差。\index{lengths of programs}

\notei{ix}
  Steele, Guy L., Jr. \emph{Common Lisp: the Language}, 2nd
  Edition. Digital Press, Bedford (MA), 1990.

\notei{5}
  Brooks, Frederick P. \emph{The Mythical Man-Month}. Addison-Wesley,
  Reading (MA), 1975, p. 16.

\notei{18}
  Abelson, Harold, and Gerald Jay Sussman, with Julie
  Sussman. \emph{Structure and Interpretation of Computer Programs}. MIT
  Press, Cambridge, 1985.

\notei{21}
  更准确地说,我们是无法仅仅用一个~\lexpr{}定义出递归函数的。不过,还是有
  办法写一个函数,让它把自己作为另外的一个参数传入,进而生成递归函数,
\begin{verbatim}
(setq fact
      #'(lambda (f n)
           (if (= n 0)
               1
               (* n (funcall f f (- n 1))))))
\end{verbatim}
\index{factorials}
  然后,再把这个函数传给另一个函数,后者会返回一个闭包,在闭包里最早
  的那个函数将会调用它自己:

\begin{verbatim}
(defun recurser (fn)
  #'(lambda (&rest args)
      (apply fn fn args)))
\end{verbatim}
\index{recurser@\texttt{recurser}}
  把~\texttt{fact} 传给~\texttt{recurser} 就会得到一个普通的阶乘函数,

\begin{verbatim}
> (funcall (recurser fact) 8)
40320
\end{verbatim}
  这个阶乘函数可以直接这样表示:
\begin{verbatim}
((lambda (f) #'(lambda (n) (funcall f f n)))
 #'(lambda (f n)
     (if (= n 0)
         1
         (* n (funcall f f (- n 1))))))
\end{verbatim}
\index{recursion 递归!without naming}
  多数~Common Lisp 用户会觉得,如果用~\texttt{label} 或
  者~\texttt{alambda} 的话,会方便很多。

\notei{23}\index{Lisp!speed of}
  Gabriel, Richard P. Performance and
  Standardization. \emph{Proceedings of the First International
    Workshop on Lisp Evolution and Standardization}, 1988,
  p. 60. 在一个实现中,通过测试~\texttt{triangle},Gabriel
  发现~``即使给~C 编译器提供手动的寄存器分配提示,Lisp 代码
  的效率仍然比该函数的~C 语言迭代版本要高出~17\%。''他的论文中还提到其
  他几个程序的速度都要优于对应的~C\index{C} 语言版本,其中一个的性能高出~42\%。

\notei{24}
  如果你希望编译所有加载了的有名函数的话,
  可以调用~\verb|compall|\index{compall@\texttt{compall}}:
\begin{verbatim}
(defun compall ()          
  (do-symbols (s)
    (when (fboundp s)
      (unless (compiled-function-p (symbol-function s))
        (print s)                   
        (compile s)))))        
\end{verbatim}
\index{fboundp@\texttt{fboundp}}
  这个函数每编译一个函数,就会打印出其函数名。

\notei{26}
  你可以调用~\texttt{(disassemble 'foo)} 来检查~inline 声明是
  否生效,这个调用会显示出函数~\texttt{foo} 某种形式的~object code。这
  同时也是检验尾递归优化是否完成的一个办法\index{tail-recursion optimization 尾递归优化!testing for 检验}。

\notei{31}
  不妨把~\texttt{nreverse} 想象成是如下那样定义的:
\begin{verbatim}
(defun our-nreverse (lst)          
  (if (null (cdr lst))
      lst
      (prog1 (nr2 lst)                            
             (setf (cdr lst) nil))))

(defun nr2 (lst)
  (let ((c (cdr lst)))          
    (prog1 (if (null (cdr c))                             
               c
               (nr2 c))
           (setf (cdr c) lst)))) 
\end{verbatim}
\index{nreverse@\texttt{nreverse}!sketch of}

\notei{43}
优秀的设计通常会把精简节约放在首位,但是程序追求短小精悍的原因却又多
了一层。程序要是写得简洁明了,你就能短时间内理解更多的代码。

大家都知道:只要能对从事的事情有大局观,设计就会变得容易一些。所以油画
家喜欢用长柄的画刷,而且时不时地退后两步端详他们的作品。所以将领宁愿把
自己置身于敌军炮火,也要在高地运筹帷幄。而程序员愿意花比小显示器更多的
钱,购置大屏幕显示器的原因也在于此。

紧凑的程序能让人的视野发挥其最大的功用。指挥官无法让一场战役在桌面
上进行,而~Lisp 却能让你把程序中的抽象逻辑浓缩在方寸之间。而且,同
时能看到的代码越多,越有可能把代码写得规整一致。

这样说并不是要求不计代价,把程序写得越短越好。要是把一个函数里所有的
换行都删掉,你也可以把它写在一行里面,但是这样做对可读性并没有帮助。紧
凑的代码要求借助抽象的机制来精简代码,而非通过文本编辑来达到目的。

不妨试想一下,如果你现在用一个比原来小一半的显示器来写程序,那会多么痛
苦啊。让你的代码缩小到原先的一半大小,一定会让编程更加得心应手。

\notei{44}
  Steele, Guy L., Jr. Debunking the ``Expensive Procedure Call'' Myth
  or, Procedural Call Implementations Considered Harmful or, LAMBDA:
  The Ultimate \textsc{Goto}.  \emph{Proceedings of the National Conference of the ACM}, 1977, p. 157.

\notei{48}
  仅供参考,下面提供了与图~\ref{fig:larger_functions_that_operate_on_lists} 
  和图~\ref{fig:doubly-recursive_list_utilities} 中对应函数更简单的实现\label{notes:simplified-implementation}。
  这些实现都相当慢~(至少慢~10\%)\index{prune@\texttt{prune}!simpler version}:

\begin{verbatim}
(defun filter (fn lst)
  (delete nil (mapcar fn lst)))

(defun filter (fn lst)
  (mapcan #'(lambda (x)
              (let ((val (funcall fn x)))
                (if val (list val))))
          lst))

(defun group (source n)
  (if (endp source)
      nil
      (let ((rest (nthcdr n source)))
        (cons (if (consp rest) (subseq source 0 n) source)
              (group rest n)))))

(defun flatten (x)
  (mapcan #'(lambda (x)
              (if (atom x) (mklist x) (flatten x)))
          x))

(defun prune (test tree)
  (if (atom tree)
      tree
      (mapcar #'(lambda (x)
                   (prune test x))
              (remove-if #'(lambda (y)
                             (and (atom y)
                                  (funcall test y)))
                         tree))))
\end{verbatim}
\index{group@\texttt{group}!simpler version}
\index{filter@\texttt{filter}!simpler version}
\index{flatten@\texttt{flatten}!simpler version}
\notei{49}
  如果像这样实现,\texttt{find2} 在走过~dotted list\index{lists!dotted} 的末端时,会产生一个错误:
\begin{verbatim} 
> (find2 #'oddp '(2 . 3)) 
>>Error: 3 is not a list.
\end{verbatim} 
  \textsc{Cltl}2 (p. 31) 提到,如果函数期望的是列表,却传给它一个~dotted
  list,这样做就是错误的。Common Lisp 实现没有义务去检测这种编程错
  误,所以有的能纠错,有的就不行。倘若函数接受序列~(sequence),这个规定
  就会显得有些含混不清。因为~dotted list 是一个~cons,而~cons 是序列。所以,
  如果严格遵守~\textsc{cltl} 的话,就应该要求
\begin{verbatim}
(find-if #'oddp '(2 . 3))
\end{verbatim} 
返回~\texttt{nil},而不是报错。因为~\texttt{find-if} 的参数应该是个序
列。

各个实现对这个表达式的反应不尽相同。有的还是会报个错,有的则会返回~\texttt{nil}。
然而,即使是遵循书中要求,在本例中表现良好的实现也会把握不住方向。比如,它们就会
认为~\texttt{(concatenate 'cons '(a . b) '(c . d))} 的结果是~\texttt{(a c . d)},
而非~\texttt{(a c)}。

本书中,凡是接受列表的\utility,要的都是正规~(proper) 的列表。用来处理
序列的\utility,都能对付~dotted list。不过,如果一个函数没有特别声明它
能处理~dotted list,你就传给它,那无异于自找麻烦。

\notei{66}

倘若我们知道每个函数的参数个数,就能写出一个~\texttt{compose},让~$f
\circ g$ 里面,$g$ 返回的多值成为~$f$ 对应的参数。在~\textsc{cltl2}
中,有这么一个新函数叫~\texttt{function-lambda-expression}\index{function-lambda-expression@\texttt{function-lambda-expression}}\index{Common Lisp!differences between versions!function-lambda-expression@\texttt{function-lambda-expression}},通过它给出
的~\lexpr{},可以看出函数最初的源代码。不过它也可能返回~\texttt{nil},这
种情况常常发生在内建函数上。我们真正需要的是一个函数,它能从传入的函数参数
得到该函数的参数列表。

\notei{73}

要让~\texttt{rfind-if}\index{rfind-if@\texttt{rfind-if}!alternate version} 搜索整个子树,可以这样定义它:

\begin{verbatim}
(defun rfind-if (fn tree) 
  (if (funcall fn tree) 
      tree
      (if (atom tree) 
          nil
          (or (rfind-if fn (car tree))
              (and (cdr tree) (rfind-if fn (cdr tree)))))))
\end{verbatim} 

传入的第一个函数参数会同时应用于原子和列表: 

\begin{verbatim}
> (rfind-if (fint #'atom #'oddp) '(2 (3 4) 5))
3
> (rfind-if (fint #'listp #'cddr) '(a (b c d e)))
(B C D E)
\end{verbatim}

\notei{95}
  McCarthy, John\index{McCarthy, John}, Paul W. Abrahams\index{Abrahams, Paul W.}, Daniel J. Edwards\index{Edwards, Daniel J.}, Timothy P. Hart\index{Hart, Timothy P.}, and Michael I. Levin\index{Levin, Michael I.}. \emph{Lisp 1.5 Programmer's Manual}, 2nd Edition. MIT Press, Cambridge, 1965, pp. 70-71.  

\notei{106}
  第~\ref{sec:when_nothing_else_will_do} 节提到,有种操作符只能写成宏,它的意思
  是:如果是用户想定义这种操作符的话,那就只能用宏。虽然宏能完成的工作,special form\index{special forms} 
  都能胜任。但是我们却无法定义新的~special form。

  special form 之所以被称为~special form,是因为它求值时是
  被当作特殊情况处理的。在解释器里,你可以把~\texttt{eval} 想象成一个庞大的
  ~\texttt{cond} 表达式:
\begin{verbatim}
(defun eval (expr env)
  (cond ...
        ((eq (car expr) 'quote) (cadr expr))
        ...
        (t (apply (symbol-function (car expr))
                  (mapcar #'(lambda (x)
                              (eval x env))
                          (cdr expr))))))
\end{verbatim}
\index{eval@\texttt{eval}!sketch of}
  缺省情况的语句处理了绝大多数表达式,在这个语句里,先从~car 里取出其指向的函数,
  然后用它对~cdr 里所有的参数求值,最后把前者应用于后者的结果作为返回值返回。
  然而,如果有表达式的样子类似于~\texttt{(quote x)}\index{quote@\texttt{quote}},那么它的处理方式就不能继续
  用这种方法了。原因在于,引用的全部意义就是在于让它的参数免于求值。因此,\texttt{eval} 
  必须有一个语句,专门对付这种引用表达式。编程语言的设计者把~special form 当成
  宪法修正案。这种条款是必不可少的,但是越少越好。\textsc{Cltl}2 在第~73 页列出
  了~Common Lisp 里所有的~special form。
  
  前面对~\texttt{eval} 的粗糙演绎不够准确,其原因在于它取得函数先于对参数求值。
  而在~Common Lisp 里,这两个操作的顺序是有意没有规定的。欲了解~Scheme 中~\texttt{eval}
  的初步实现,可以参考~Abelson and Sussman, p. 299。

\notei{115}
  简单说,如果编写一个\utility{}函数能帮助省下它自身的代码量,那么它的存在就是合理的。
  但如果~\utility{}是宏的话,标准应该更高一点。看懂宏调用比看懂函数调用要费劲些,因
  为它们有可能会违反~Lisp 的求值规则。在~Common Lisp 里,求值规则是这样规定的:
  表达式的值的计算方法如下,car 里指定的是函数名,而~cdr 中指定了参数列表,
  参数列表求值的顺序是从左至右,然后将函数应用于参数列表,所得到的结果则作为
  表达式的值。由于函数调用都遵循着这个规则,所以弄懂~\texttt{find2} 并不比
  理解~\texttt{find-books} 更困难~(第~\pageref{func:find-books} 页)。

  然而,宏在很多情况下并不遵守~Lisp 的这个求值规范。(要是有宏遵守的话,
  你或许当初就应该用函数来实现它。) 原则上来讲,每个宏都定义了它自己的一套求
  值规则,而~reader 只有在读取了宏的定义之后,才能知道这套规则到底是怎
  么样的。总而言之,一个宏的存在是不是有意义,要视其可读性而定,同时它
  节省的代码量必须要远大于其自身的代码量。\index{macros 宏!justification of}


\notei{126}

  和书中其它几个~\texttt{for} 一样,
  图~\ref{fig:avoiding_capture_with_closure} 中定义的~\texttt{for},
  对~\texttt{do} 表达式里的~initform 也有着相同的假设,即这个~initform
  是从左到右求值的,只有满足这个要求,\texttt{for} 的定义才是正确的。
  \textsc{Cltl}2 (p. 165) 提到这个说法对~stepform 成立,但是
  对~initform 却只字未提。

  把这个问题当成一次疏忽大意的结果也未尝不可。一般来说,如果有操作的顺
  序没有被指定,\textsc{cltl} 应该会特意指出来的。而且,也没有理由不
  把~\texttt{do} 的~initform 的求值顺序确定下来。因为~\texttt{let} 的求
  值是从左到右的,而~\texttt{do} 自己的~stepform 的求值顺序也是一样的。

\notei{128}

  Common Lisp 的~\texttt{gentemp}\index{gentemp@\texttt{gentemp}} 和~\texttt{gensym} 差不多,其不同之处
  在于,前者会~intern 它生成的符号。和~\texttt{gensym} 一样,
  \texttt{gentemp} 在内部维护着一个计数器,这个计数器在构造~print
  name 的时候会用到。如果~\texttt{gentemp} 在生成某个符号的时候,当前
  的~package 里已经有了这个符号,它就会递增计数器,然后再试一次:

\begin{verbatim}
> (gentemp)
T1
> (setq t2 1)
1
> (gentemp)
T3
\end{verbatim}

这样,就保证了新创建的符号是独一无二的。然而,也可以想见,
\texttt{gentemp} 生成的符号还是无法完全免于名字冲突。尽
管~\texttt{gentemp} 有能力确保生成的符号是迄今为止还没有出现过,但是它
无法预见将来会出现些什么符号。既然~\texttt{gensyms} 能很好地完成工作,
而且绝大多数时候是安全的,那么为什么还要用~\texttt{gentemp} 呢?实际
上,对于编写宏来说,\texttt{gentemp} 唯一的优点就是它产生的符号可以写出
去,之后再读进来,在这种情景下,这些符号就没有必要是唯一的了。

\notei{131}

对于~Scheme 来说,由于它采用的是统一的名字空间\index{name-spaces},函数的名字捕捉会带来更
多的麻烦。在过去,Scheme 标准一直没有对宏定义方法的给出一个正式的规定\index{Scheme!macros in},
这个状态一直持续到~1991 年才有所改观。当前,Scheme 提供的卫生
宏~(hygienic macro\index{hygienic macros}\index{macros 宏!hygienic}) 和~\texttt{defmacro} 之间的区别相当大。欲知详情,或
是想了解近来相关的参考文献,可见最新的~Scheme report\footnote{译者注:
迄今为止,最新的~Scheme report 为
~\href{http://www.r6rs.org/}{\textsc{r6rs}}。}。

\notei{137}

Miller, Molly M., and Eric Benson. \emph{Lisp Style and Design}.  Digital
Press, Bedford (MA), 1990, p. 86.  

\notei{158}

如果不写~\texttt{mvpsetq},改成定义~\texttt{values} 的逆操作\index{values@\texttt{values}!inversion for 逆操作}可能会让代码更简洁些。
这样,就不再是
\begin{verbatim}
(mvpsetq (w x) (values y z) ...)
\end{verbatim} 
我们会这样写
\begin{verbatim}
(psetf (values w x) (values y z) ...)
\end{verbatim}

如果有了~\texttt{values} 的逆,那么~\texttt{multiple-value-setq} 也可以
淘汰了。不过,遗憾的是,由于~Common Lisp 的固有特性,无法定义这样的逆操
作。\texttt{get-setf-method} 不可能返回一个以上的~store variable,因此
可以推知,就算~\texttt{get-setf-method} 真的这样做的话,\texttt{psetf}
的展开函数也不会知道该如何应对这种情况。

\notei{180}

\texttt{setf} 教给我们的其中一件事,就是有的宏能完成相当数量的计算工
作,同时让源代码变得清晰可读。最终,\texttt{setf} 可能是被用来进行
带~assertion 编程的宏之一。

举例来说,如果能有个宏~\texttt{insist},它可能会对我们有所帮助。它接受
的表达式的模样像这样:\texttt{($predicate$ . $arguments$)},如果表达式
的值不为真,宏就会想办法让它成为真。\texttt{setf} 必须被告知如何进行引
用的逆操作,同样的道理,这个宏一样得知道如何把表达式的值变成真。一般情
况下,这种宏调用可能会最终调用~Prolog。

\notei{198}

Gelernter, David H., and Suresh Jagannathan. \emph{Programming Linguistics}.
MIT Press, Cambridge, 1990, p. 305.

\notei{199}

Norvig, Peter. \emph{Paradigms of Artificial Intelligence Programming: Case
Studies in Common Lisp}. Morgan Kaufmann, San Mateo (CA), 1992, p. 856.

\notei{213}

这个叫~\texttt{least-negative-normalized-double-float} 的常量和它三个兄弟
的名字长度在~Common Lisp 里排名并列第一,均达~38 个字符。名字最长的操作符
是~\texttt{get-setf-method-multiple-value},共计~30 个字符。\index{Common Lisp!long names in}下面的表达
式将会返回一个列表,列表里把当前包中所有可见符号按照名字从长到短排了一遍:
     
\begin{verbatim}
(let ((syms nil))
  (do-symbols (s)
    (push s syms))
  (sort syms
        #'(lambda (x y)
             (> (length (symbol-name x))
                (length (symbol-name y))))))
\end{verbatim}


\notei{217}
\label{notes:difference-null-environment}
跟据~\textsc{cltl}2,宏展开函数的定义环境\index{environment!of macro
expanders 宏展开器}\index{macros 宏!environment of expander}应该就在~\texttt{defmacro} 表达式出现的地方。这样的
话,我们就可以把~\texttt{propmacro}\index{propmacro@\texttt{propmacro}!alternative definition} 定义得更简洁:

\begin{verbatim}
(defmacro propmacro (propname)
  `(defmacro ,propname (obj)
     `(get ,obj ',propname)))
\end{verbatim}

不过~\textsc{cltl}2 并没有点明,究竟最初传给~\texttt{propname} 的~form
是不是词法环境的一部分,其中,词法环境是里面的~\texttt{defmacro} 发生的地
方。照道理说,如果是用~\texttt{(propmacro color)} 定义
的~\texttt{color},那么它就等价于:

\begin{verbatim}
(let ((propname 'color))
  (defmacro color (obj)
    `(get ,obj ',propname)))
\end{verbatim}
或是
\begin{verbatim}
(let ((propname 'color))
  (defmacro color (obj)
    (list 'get obj (list 'quote propname))))
\end{verbatim}

可是,至少有几个~\textsc{cltl2} 实现,在它们里面新版
的~\texttt{propname} 无法正常工作。在~\textsc{cltl}1 里面则认为宏的展开
函数是在一个空的词法环境\index{environment!null 空}里定义的。所以,为了
最大的可移植性,不管如何,宏定义应该避免使用外部的环境。

\notei{238}

有时,会把类似~\texttt{match} 的函数说成一种合一运算~(unification)\index{unification 合一}
\footnote{译者注:\emph{合一}是符号逻辑中的概念,常用在谓词演算的归结
中。简单说,合一就是:为多个~(通常为两个) 命题中的自由变量找到对应的
值~(或者代换关系),令命题间的匹配关系得以满足,从而把几个命题合成一个新
的命题。可以说,合一是一种广义的模式匹配操作。}。但这种说法不是很准确。
比如说,\texttt{match} 可以把~\texttt{(f ?x)} 和~\texttt{?x} 匹配起来,
但是两个表达式是不可合一的。

如果需要合一的详细描述,可见: Nilsson, Nils J. \emph{Problem-Solving
Methods in Artificial Intelligence}\footnote{译者注:亦可参考:Nilsson, Nils J. 
\emph{Artificial Intelligence: A New Synthesis}. Morgan Kaufmann, 1998, pp. 253-255. }. McGraw-Hill,
New York, 1971, pp. 175-178.

\notei{244-1}

并不是一定要把未绑定的变量用~gensym 来表示,在运行时调
用~\texttt{gensym?} 也不是唯一的选择。
图~\ref{fig:fast_matching_operator} 和
图~\ref{fig:fast_matching_operator_continued} 中展开生成的代码也可以这
样写,让程序记住那些绑定代码已经生成了的变量。不过,如果要这样实现的
话,代码很可能需要从把逻辑从里到外颠倒一下:即不再在递归返回的时候生成
展开代码,而是要在递归过程层层递进的时候累积代码。

\notei{244-2}

在~\texttt{if-match} 匹配模式中,类似~\texttt{?x} 的符号代表的总是新变
量,这和~\texttt{let} 绑定语句中~car 指向的符号情况一样。因此,
尽管~Lisp 变量可以用在匹配模式里,但是外部查询中的匹配变量却不行。
原因是,虽然你可以用一样的\emph{符号},但是它代表是个不一样的新变量。
要检验两个列表的第一个元素是不是一样的,用下面的代码是无法完成任务的:
   
\begin{verbatim}
(if-match (?x . ?rest1) lst1
    (if-match (?x . ?rest2) lst2
        ?x))
\end{verbatim}

在本例中,第二个~\texttt{?x} 是个新变量。如
果~\texttt{lst1} 和~\texttt{lst2} 都至少有一个元素,那么上面的表达式会总是返回~\texttt{lst2} 的~car。

然而,由于你可以在~\texttt{if-match} 中的匹配模式里
使用~(没有~\texttt{?} 过的) Lisp 变量,所以下面的代码可以带给你期望的效果:
\begin{verbatim}
(if-match (?x . ?rest1) lst1
    (let ((x ?x))
      (if-match (x . ?rest2) lst2
          ?x)))
\end{verbatim}

这个限制和解决的办法同样也适用于第~\ref{chap:a_query_compiler} 章和
第~\ref{chap:prolog} 章里定义的~\texttt{with-answer} 和~\texttt{with-inference} 宏。

\notei{254}

``未绑定''的模式匹配变量将是~\texttt{nil},万一这个有所不妥,
你可以把这些匹配变量绑定到专门的~gensym 上,也就是:
先~\texttt{(defconstant unbound (gensym))},再把~\texttt{with-answer} 里的
\begin{verbatim}
`(,v (binding ',v ,binds)))
\end{verbatim}
这一行换成:
\begin{verbatim}
`(,v (aif2 (binding ',v ,binds) it unbound))
\end{verbatim}

\notei{258}
  Scheme 由~Guy L. Steele Jr.\index{Steele, Guy Lewis Jr.} 和~Gerald J. Sussman 在~1975 年发明。
  该语言的定义文件是:Clinger, William\index{Clinger, William}, 和~Jonathan A. Rees\index{Rees, Jonathan A.} (及他人编辑)。
  \emph{Revised$^4$ Report on the Algorithmic Language Scheme}. 1991.  

  这份报告以及各种~Scheme 实现在本书付梓之时,可由匿
  名~\textsc{ftp} 在~\texttt{altdorf.ai.mit.edu:pub} 下载到。

\notei{266}
  下面为第~\ref{chap:macro-defining_macros} 章中介绍的技术再补充个例子,
  它由~\texttt{=defun} 定义中的~\texttt{defmacro} 模板衍生而来:
\begin{verbatim}
(defmacro fun (x)
  `(=fun *cont* ,x))

(defmacro fun (x)
  (let ((fn '=fun))
    `(,fn *cont* ,x)))

`(defmacro ,name ,parms
   (let ((fn ',f))
     `(,fn *cont* ,,@parms)))

`(defmacro ,name ,parms
   `(,',f *cont* ,,@parms))
\end{verbatim}
\index{backquote@backquote (\texttt{`}) \bq!nested 嵌套}

\notei{267}
  如果你希望在 toplevel 看到多重返回值的话,可以改成这样说:
\begin{verbatim}
(setq *cont*
      #'(lambda (&rest args)
          (if (cdr args) args (car args))))
\end{verbatim}

\notei{273-1}
  这个例子基于另一个例子,它来自:Wand, Mitchell.\index{Wand, Mitchell}
  Continuation-Based Program Transformation Strategies. \emph{JACM} 27, 1
  (January 1980), pp. 166.

\notei{273-2}

  在这本书里有一个能把~Scheme 代码转换到~continuation-passing style 的程序:
  Steele, Guy L., Jr.\index{Steele, Guy Lewis Jr.} \emph{\textsc{LAMBDA}: The Ultimate
  Declarative}. MIT Artificial Intelligence Memo 379, November 1976,
  pp. 30-38.  

\notei{292}
  下面的~\texttt{choose} 和~\texttt{fail} 实现在~T 语言\index{T}里会更简
  洁。T 是~Scheme 的一种方言,它提供了~\texttt{push} 和~\texttt{pop} 操
  作,而且允许在非~toplevel 的上下文里进行定义:

\begin{verbatim}
(define *paths* ())
(define failsym '@)

(define (choose choices)
  (if (null? choices)
      (fail)
      (call-with-current-continuation
        (lambda (cc)
          (push *paths*
                (lambda () (cc (choose (cdr choices)))))
          (car choices)))))

(call-with-current-continuation
  (lambda (cc)
    (define (fail)
      (if (null? *paths*)
          (cc failsym)
          ((pop *paths*))))))
\end{verbatim}
\label{note:func:choice-implemented-in-t}
\index{true-choose@\texttt{true-choose}!breadth-first version!T implementation}
  欲知~T 语言的具体情况,可参考:Rees, Jonathan A.\index{Rees, Jonathan A.}, Norman
  I. Adams\index{Adams, Norman I.}, and James R.  Meehan\index{Meehan, James R.}.  \emph{The
    T Manual}, 5th Edition. Yale University Computer Science
  Department, New Haven, 1988.

  T 语言参考手册,以及~T 语言本身在本书付梓时,可经由匿
  名~\textsc{ftp} 在~\texttt{hing.lcs.mit.edu:pub/t3.1} 下载到。

\notei{293}
  Floyd, Robert W. Nondeterministic
  Algorithms. \emph{JACM} 14, 4 (October 1967), pp. 636-644.  

\notei{298}

  第~\ref{chap:continuations} 章定义的~continuation-passing 宏相当依赖
  尾递归调用的优化。如果没有这种优化,这些宏对规模较大的问题就无能为力
  了。比如说,在本书付印之时,在没有尾递归优化的情况下,几乎没有计算机
  有足够的内存能用第~\ref{chap:prolog} 章里实现的~Prolog 运行~zebra
  benchmark\index{zebra benchmark}。\index{tail-recursion optimization 尾递归优化!testing for 检验} (警告:有的~Lisp 在耗尽栈空间的时候会崩溃。)\index{stacks 栈!overflow of}
  
\notei{303}
  通过显式地避免循环路径,来定义一个深度优先的正确 \emph{choose} 是完全可行的。
  下面是一个~T 语言的实现:
\begin{verbatim}
(define *paths* ())
(define failsym '@)
(define *choice-pts* (make-symbol-table))
(define-syntax (true-choose choices)
  `(choose-fn ,choices ',(generate-symbol t)))

(define (choose-fn choices tag)
  (if (null? choices)
      (fail)
      (call-with-current-continuation
        (lambda (cc)
          (push *paths*
                (lambda () (cc (choose-fn (cdr choices)
                                          tag))))
          (if (mem equal? (car choices)
                          (table-entry *choice-pts* tag))
              (fail)
              (car (push (table-entry *choice-pts* tag)
                         (car choices))))))))
\end{verbatim}
\label{func:true-choice}
\index{true-choose@\texttt{true-choose}!depth-first version}
  在这个版本里面,\texttt{true-choose} 成了一个宏。(T 语言
  的~\texttt{define-syntax} 和~\texttt{defmacro} 很相似,只不过宏的名字
  是放在参数列表的~car 里。) 这个宏将展开成对~\texttt{choose-fn} 的一次
  调用,\texttt{choose-fn} 与
  图~\ref{fig:scheme_implementation_of_choose_and_fail} 中定义的深度优
  先~\texttt{choose} 差不多,区别在于前者另外接受一个~\texttt{tag} 参数
  以区别出选择点。每一个~\texttt{true-choose} 的返回值都被保存在全局的
  哈希表~\texttt{*choice-pts*} 中。如果有~\texttt{true-choose} 准备返回
  一个以前返回过的值,那么它就会失败。不过没有必要修改~\texttt{fail} 本
  身。我们可以借用一下第~\ref{note:func:choice-implemented-in-t} 页上定
  义的~\texttt{fail}。
  
  这个实现做了一个假设,即路径的长度都是有限的。比如说,这让
  图~\ref{fig:nondeterministic_search} 中定义的~\texttt{path} 能在
  图~\ref{fig:a_directed_graph_with_a_loop} 中的有向图,中找到
  从~\texttt{a} 走~\texttt{e} 到的路径~(不过
  图~\ref{fig:a_directed_graph_with_a_loop} 中的也可以不是有向图)。但是上面
  定义的~\texttt{true-choose} 对有无穷搜索空间的问题是无能为力的:
\begin{verbatim}
(define (guess x)
  (guess-iter x 0))

(define (guess-iter x g)
  (if (= x g)
      g
      (guess-iter x (+ g (true-choose '(-1 0 1))))))
\end{verbatim}
  
  使用上面定义的~\texttt{true-choose},\texttt{(guess $n$)} 只在~$n$ 为
  非负数的时候才会终止。
  
  我们定义正确~\emph{choose} 的方式取决于如何界定选择点。在这个版
  本中,把所有形式上调用~\texttt{true-choose} 的地方都认作选择点。
  这样做对于有的应用场合来说可能会有些局限性。比如说,如果~\texttt{two-number} 
  (第~\pageref{fig:choice_in_a_subroutine} 页) 使用的是这个版本
  的~\emph{choose},它就永远不会重复返回相同的一对数字,即使调用它的
  是不同的函数。这样的行为有可能是我们想要的,也有可能不是,这要视情况
  而定。
  
  请注意,这个版本的实现只是用在编译后的代码里的。在解释的代码中,
  宏调用会被重复展开,每次都会生成一个新~gensym 过的~tag。

\notei{305}
  Woods, William A. Transition Network Grammars for Natural Language
  Analysis.  \emph{CACM} 3, 10 (October 1970), pp. 591-606.

\notei{312}
  最初的~\textsc{atn} 系统里的操作符是能在子网络里控制和修改栈上的寄存
  器的。要加入这个特性的并不难,但是我们还有一个更通用的解决方案:就是
  在附于转移弧上的代码中直接加入一个~lambda 表达式,让这个表达式能应用
  于寄存器栈。举个例子,如果节点~\texttt{mods}
  (第~\pageref{fig:sub-network_for_strings_of_modifiers} 页) 把如下代码
  加入出弧上的代码体的话,
\begin{verbatim}
(defnode mods
  (cat n mods/n
    ((lambda (regs)
       (append (butlast regs) (setr a 1 (last regs)))))
    (setr mods *)))
\end{verbatim}
  那么状态顺着这条弧转移的话~(不管走多远),都会把寄存器~\texttt{a} 的最上面
  的实例~(也就是在~\textsc{atn} 的顶层网络中遍历时,可见的~\texttt{a}
  实例) 设置成~\texttt{1}。

\notei{323}
  
  如果有必要的话,可以很容易地对这个~Prolog 加以修改,让它能利用现有的
  事实数据库。方法是把~\emph{choose} 嵌到~\texttt{prove}
  (第~\pageref{fig:compilation_of_queries}) 里面:
\begin{verbatim}
(=defun prove (query binds)
  (choose
     (choose-bind b2 (lookup (car query) (cdr query) binds)
       (=values b2))
     (choose-bind r *rules*
       (=funcall r query binds))))
\end{verbatim}

\notei{325}
  欲迅速判断一个查询是否会有与之匹配的结果,可以使用下面的宏:

\begin{verbatim}
(defmacro check (expr)
  `(block nil
     (with-inference ,expr
       (return t))))
\end{verbatim}

\notei{344}
  本节中用到的例子是从这本书中的例子转换而来的:
  Sterling, Leon\index{Sterlin, Leon}, and Ehud Shapiro\index{Shapiro, Ehud}. \emph{The Art of Prolog: Advanced
  Programming Techniques}. MIT Press, Cambridge, 1986。

\notei{349}
  Lisp 中的底层概念没有一个明确的称呼,这可能会严重阻碍这门语言
  的推广。要是有人喊出这样的口号,说“我们要用 C++\index{C++} 因为想做面向对象的编
  程”,那还说得过去。但如果要说“我们要用 Lisp,因为要做 Lisp 编程”,这听上去
  就没那么理直气壮。在管事的人听起来,这话像是循环论证。在他们眼中,
  如果~Lisp 很强大,那么其价值就应该来自某一个简单易懂的理念。多年来,我们试着
  说服他们,但是收效甚微。长久以来,Lisp 被说成一门“处理列表的
  语言”\index{list processing},和用来做“符号计算”\index{symbolic computation 符号计算}的语言,最近又有种说
  法,称之为“动态语言\index{dynamic languages  动态语言}”。但是这些名字中,没有一个能体现出 Lisp 的精
  髓,哪怕有一点点沾边。大学里有关编程语言的教科书则把这些名号论斤作
  价,拆散零售,这些论断极大地误导了读者。

  所有把 Lisp 概括成一个词的尝试终将以失败告终,因为 Lisp 的能力源于它
  的至少五个或六个特性。在“Lisp 给了我们什么”这个问题前,或许我们应该认
  一回输,承认只有 \emph{Lisp} 才是唯一准确的答案。

\notei{352}
  出于效率的考虑,如果第二个参数函数认为序列中有元素是相等
  的,那么 \texttt{sort} 是不会去保持它们原来次序的。比如说,一个正常
  的 Common Lisp 实现可能会这样做:
\begin{verbatim}
> (let ((v #((2 . a) (3 . b) (1 . c) (1 . d))))
    (sort (copy-seq v) #'< :key #'car))
#((1 . D) (1 . C) (2 . A) (3 . B))
\end{verbatim}
  可以注意到,其中最前面两个元素的相对次序被颠倒了。 内置
  的 \texttt{stable-sort}\index{stable-sort@\texttt{stable-sort}} 可以在排序的同时,保持元素原来的先后次序:
\begin{verbatim}
> (let ((v #((2 . a) (3 . b) (1 . c) (1 . d))))
    (stable-sort (copy-seq v) #'< :key #'car))
#((1 . C) (1 . D) (2 . A) (3 . B))
\end{verbatim}
  把 \texttt{sort} 当成 \texttt{stable-sort} 使用是一种常见错误。另一个
  误区是认为 \texttt{sort} 是非破坏性的。事实上,不管
  是 \texttt{sort} 还是 \texttt{stable-sort} 都有可能修改它们的排序对象。
  如果你不希望传入的序列发生变动,那你排序的应该是个拷贝。不
  过,在 \verb|get-ancestors| 里调用 \verb|stable-sort| 是安全的,因为
  被排序的列表是新生成的。
\end{notes}

%%% Local Variables:
%%% coding: utf-8
%%% mode: latex
%%% TeX-master: "onlisp-cn"
%%% End:
