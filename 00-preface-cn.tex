%!TEX encoding = UTF-8 Unicode
% $Id: 00-preface-cn.tex 3 2009-08-07 06:58:03Z binghe $

\chapter*{译者序}
\label{chap:preface-cn}

《On Lisp》不是一本~Lisp 的入门教材,它更适合读过%
\href{http://www.paulgraham.com/acl.html}{《ANSI Common Lisp》}%
或者\href{http://gigamonkeys.com/book/}{《Practical Common Lisp》}%
的~Lisp 学习者。它对~Lisp 宏本身及其使用做了非常全面的说明,同时自底向
上的编程思想贯穿全书,这也是本书得名的原因,即,基于~Lisp,扩展~Lisp。

原作者~\href{http://www.paulgraham.com/}{Paul Graham} 同时也
是《ANSI Common Lisp》一书的作者。

《On Lisp》成书早在~1994 年~ANSI Common Lisp 标准发布以前,书中使用了许多古
老的~Lisp 操作符,其中一些代码已经无法在最新的~Common Lisp 平台上执行了。
所以译文里所有的源代码都被改成了符合现行~Common Lisp 标准的形式,凡译者
修改过的地方都会以脚注的形式注明。

我要特别感谢来自~AMD/ATI 的~Kov Chai\footnote{
\href{mailto:tchaikov@gmail.com}{\texttt{tchaikov@gmail.com}}}
同学,他独立翻译了第~\ref{chap:returning_functions},
\ref{chap:functions_as_representation},
\ref{chap:nondeterminism},
\ref{chap:parsing_with_atns},
\ref{chap:object-oriented_lisp} 章及附录,并对全书进行了细致的校对。
另外~Kov Chai 还主导了本书的~\LaTeX~排版工作。

感谢~Yufei Chen\footnote{
\href{mailto:cyfdecyf@gmail.com}{\texttt{cyfdecyf@gmail.com}}}
同学提供改进排版的补丁。他还参与了第~\ref{chap:multiple_processes}
章的翻译工作。

Mathematical Systems, Inc. 的~Lisp 程序员~Jianshi Huang\footnote{
\href{mailto:jianshi.huang@gmail.com}{\texttt{jianshi.huang@gmail.com}}}
同学是我最初翻译本书时的合作者,他翻译了第~\ref{chap:prolog} 章,并
初步校对了本书前三章。

\vfill
\hfill \emph{Chun TIAN (binghe)}\footnote{
  \href{mailto:binghe.lisp@gmail.com}{\texttt{binghe.lisp@gmail.com}}}

\hfill \emph{NetEase.com, Inc.}
\vfill


%%% Local Variables:
%%% coding: utf-8
%%% mode: latex
%%% TeX-master: "onlisp-cn"
%%% End:
