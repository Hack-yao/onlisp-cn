%!TEX encoding = UTF-8 Unicode
% $Id: 18-destructuring.tex 18 2014-03-12 22:35:24Z binghe $

\chapter{解构}
\label{chap:destructuring}

解构~(destructuring) 是赋值的一般形式。操作符~\texttt{setq} 和~\texttt{setf} 的赋值对象只是
独立的变量。而解构把赋值和访问操作合二为一:在这里,我们不再只是把单个变量作为第一个参数,而是给出一个关于变量的模式,
在这个模式中,赋给每个变量的值将来自某个结构中对应的位置。

\section{列表上的解构}
\label{destructuring_on_lists}
\index{destructuring!on lists}

从~\textsc{cltl}2 开始,Common Lisp 包括了一个名为~\texttt{destructuring-bind}
的新宏。这个宏在第~\ref{chap:macros} 章里简单介绍过。
这里将更仔细地了解它。假设~\texttt{lst} 是一个三元素列表,而我们想要绑定~\texttt{x}
到第一个元素,\texttt{y} 到第二个,\texttt{z} 到第三个。在原始~\textsc{cltl}1
的~Common Lisp 里,只能这样表达:
\begin{lstlisting}
(let ((x (first lst))
      (y (second lst))
      (z (third lst)))
  ...)
\end{lstlisting}
借助新宏我们只需说:
\begin{lstlisting}
(destructuring-bind (x y z) lst
  ...)
\end{lstlisting}
这样处理,既短小,又清楚。
读者对于视觉形象的感受力比单纯的文字要敏锐很多\index{visual aspects of source code 源代码在视觉感受上的一面}。
使用后一种形式,\texttt{x},\texttt{y} 和~\texttt{z} 之间的关系可以一览无余;
而在前一种形式下,我们必须稍作思考才看得出来。

如果这样简单的情形都能通过使用解构而变得更清晰,试想一下它在更复杂情况下会带来什么样的改观吧。
\texttt{destrucuring-bind} 的第一个参数可以是任意复杂的一棵树。想象
\begin{lstlisting}
(destructuring-bind ((first last) (month day year) . notes)
                    birthday
  ...)
\end{lstlisting}
如果用~\texttt{let} 和列表访问函数来写将会是什么样子。这引出了另一个要点:
解构使得程序更容易写也更容易读。

解构在~\textsc{cltl}1 的~Common Lisp 里确实也有过。如果上例中的模式看起来眼熟
的话,那是因为它们和宏的参数列表具有相同的形式。事实上,\texttt{destructuring}
\emph{就是} 就是用来处理宏参数列表的代码,只不过现在拿出来单卖了。任何可以放进
宏参数列表里的东西,你都可以把它置于这个匹配模式中,不过有个无关紧要的例外~(那个~\texttt{\&environment}
关键字)。 

建立各种绑定\emph{总的来说}是一个很有吸引力的想法。接下来的几个小节会介绍这个主题的几个变化。

\section{其他结构}
\label{sec:other_structures}

没有理由把解构仅限于列表。解构同样适用于各种复杂对象。本节展示如何编写用于其他类型对象的
类似~\texttt{destructuring-bind} 的宏。

\begin{figure}
\begin{lstlisting}
(defmacro dbind (pat seq &body body)
  (let ((gseq (gensym)))
    `(let ((,gseq ,seq))
       ,(dbind-ex (destruc pat gseq #'atom) body))))

(defun destruc (pat seq &optional (atom? #'atom) (n 0))
  (if (null pat)
      nil
      (let ((rest (cond ((funcall atom? pat) pat)
                        ((eq (car pat) '&rest) (cadr pat))
                        ((eq (car pat) '&body) (cadr pat))
                        (t nil))))
        (if rest
            `((,rest (subseq ,seq ,n)))
            (let ((p (car pat))
                  (rec (destruc (cdr pat) seq atom? (1+ n))))
              (if (funcall atom? p)
                  (cons `(,p (elt ,seq ,n))
                        rec)
                  (let ((var (gensym)))
                    (cons (cons `(,var (elt ,seq ,n))
                                (destruc p var atom?))
                          rec))))))))

(defun dbind-ex (binds body)
  (if (null binds)
      `(progn ,@body)
      `(let ,(mapcar #'(lambda (b)
                         (if (consp (car b))
                             (car b)
                             b))
                     binds)
         ,(dbind-ex (mapcan #'(lambda (b)
                                (if (consp (car b))
                                    (cdr b)))
                            binds)
                    body))))
\end{lstlisting}
  \caption{通用序列解构操作符}
  \label{fig:general_sequence_destructuring_operator}
  \index{dbind@\texttt{dbind}}
  \index{destruc@\texttt{destruc}}
\end{figure}

下一步,自然是去处理一般性的序列\index{destructuring!on sequences}。
图~\ref{fig:general_sequence_destructuring_operator} 中定义了一个名为~\texttt{dbind}
的宏,它和~\texttt{destrucuring-bind} 类似,不过可以用在任何种类的序列上。第二个参数可以是
列表,向量\index{vectors 向量!matching}或者它们的任意组合:
\begin{lstlisting}
> (dbind (a b c) #(1 2 3)
    (list a b c))
(1 2 3)
> (dbind (a (b c) d) '(1 #(2 3) 4)
    (list a b c d))
(1 2 3 4)
> (dbind (a (b . c) &rest d) '(1 "fribble" 2 3 4)
    (list a b c d))
(1 #\f "ribble" (2 3 4))
\end{lstlisting}
\verb|#(|\index{\#(@\texttt{\#(}} 读取宏用于表示向量,而~\verb|#\|\index{\#\textbacksplash\@\texttt{\#\textbackslash}} 则用于表示字符。由于
~\verb|"abc" = #(#\a #\b #\c)|,所以
~\texttt{"fribble"} 的第一个元素是字符~\verb|#f|\index{strings 字符串!as vectors}。为了简单起见,
\texttt{dbind} 只支持~\verb|&rest| 和~\verb|&body| 关键字。

和迄今为止见过的大多数宏相比,\texttt{dbind} 俨然是个庞然大物。这个宏的实现之所以值得好好研究一番,原因不仅仅是为了
理解它的工作方式,更是为了它能给我们上一课,课的内容对于~Lisp 编程是通用的。
正如第~\ref{sec:interactive_programming} 节提到的,我们在编写~Lisp 程序时,可以有意识地让它们更易于测试。\index{macros 宏!clarity 清晰}
在多数代码里,我们必须要权衡这一诉求和代码速度上的需求。
幸运的是,如第~\ref{sec:macro_style} 节所述,速度对于展开器代码来说不是那么要紧。
当编写用来生成宏展开式的代码时,我们可以让自己放轻松一些。\texttt{dbind} 的展开式由两个
函数生成,\texttt{destruc} 和~\texttt{dbind-ex}。也许它们两个可以被合并成一个函数,
一步到位。但是何苦呢?作为两个独立的函数,它们将更容易测试。为什么要牺牲这个优势,
换来我们并不需要的速度呢?

第一个函数是~\texttt{destruc},它遍历匹配模式,将每个变量和运行期对应对象的位置关联在一起:
\begin{lstlisting}
> (destruc '(a b c) 'seq #'atom)
((A (ELT SEQ 0)) (B (ELT SEQ 1)) (C (ELT SEQ 2)))
\end{lstlisting}
可选的第三个参数是个谓词,它用来把模式的结构和模式的内容区分开。

为了使访问更有效率,一个新的变量~(生成符号) 将被绑定到每个子序列上:
\begin{lstlisting}
> (destruc '(a (b . c) &rest d) 'seq)
((A (ELT SEQ 0))
 ((#:G2 (ELT SEQ 1)) (B (ELT #:G2 0)) (C (SUBSEQ #:G2 1)))
 (D (SUBSEQ SEQ 2)))
\end{lstlisting}

\texttt{destruc} 的输出被发送给~\texttt{dbind-ex},后者被用来生成宏展开
代码。它将~\texttt{destruc} 产生的树转化成一系列嵌套的~\texttt{let}:
\begin{lstlisting}
> (dbind-ex (destruc '(a (b . c) &rest d) 'seq) '(body))
(LET ((A (ELT SEQ 0))
      (#:G4 (ELT SEQ 1))
      (D (SUBSEQ SEQ 2)))
  (LET ((B (ELT #:G4 0))
        (C (SUBSEQ #:G4 1)))
    (PROGN BODY)))
\end{lstlisting}

注意到~\texttt{dbind},和~\texttt{destructuring-bind} 一样,假设它将发现所有它寻找
的列表结构。最后剩下的变量并不是简单地绑定到~\texttt{nil}\index{multiple-value-bind@\texttt{multiple-value-bind}!leftover parameters \texttt{nil}},
就像~\texttt{multiple-value-bind} 那样。如果运行期给出的序列里没有包含所有期待的元素,
解构操作符将产生一个错误:
\begin{lstlisting}
> (dbind (a b c) (list 1 2))
>>Error: 2 is not a valid index for the sequence (1 2)
\end{lstlisting}

\begin{figure}
\begin{lstlisting}
(defmacro with-matrix (pats ar &body body)
  (let ((gar (gensym)))
    `(let ((,gar ,ar))
       (let ,(let ((row -1))
               (mapcan
                 #'(lambda (pat)
                     (incf row)
                     (let ((col -1))
                       (mapcar #'(lambda (p)
                                   `(,p (aref ,gar
                                              ,row
                                              ,(incf col))))
                               pat)))
                 pats))
         ,@body))))

(defmacro with-array (pat ar &body body)
  (let ((gar (gensym)))
    `(let ((,gar ,ar))
       (let ,(mapcar #'(lambda (p)
                         `(,(car p) (aref ,gar ,@(cdr p))))
                     pat)
         ,@body))))
\end{lstlisting}
  \caption{数组上的解构}
  \label{fig:destructuring_on_arrays}
  \index{with-matrix@\texttt{with-matrix}}
  \index{with-array@\texttt{with-array}}
\end{figure}

其他有内部结构的对象该怎么处理呢?通常还有数组,它和向量的区别在于其维数可以大于一。
如果我们为数组定义解构宏\index{destructuring!on arrays},我们怎样表达匹配模式呢?对于两维数组,用列表还是比较实际的。
图~\ref{fig:destructuring_on_arrays}\footnote{译者注:这里稍微修改了一下原书的代码,
  原书中没有定义~\texttt{col} 变量就直接使用了~\texttt{(setq col -1)},
  这里仿照~\texttt{row} 的处理方法用~\texttt{let} 建立了一个~\texttt{col}
  的局部绑定。} 含有一个宏,\texttt{with-matrix},用于解构两维数组。
\begin{lstlisting}
> (setq ar (make-array '(3 3)))
#<Simple-Array T (3 3) C2D39E>
> (for (r 0 2)
    (for (c 0 2)
      (setf (aref ar r c) (+ (* r 10) c))))
NIL
> (with-matrix ((a b c)
                (d e f)
                (g h i)) ar
    (list a b c d e f g h i))
(0 1 2 10 11 12 20 21 22)
\end{lstlisting}

对于大型数组,或者维数是~3 或更高的数组来说,我们就需要另辟奚径。我们不大可能把
一个大数组里的每一个元素都绑定到变量上。将匹配模式做成数组的稀疏表达将会更实际一些\pozhehao{}只包含用于少数
元素的变量,加上用来标识它们的坐标。图~\ref{fig:destructuring_on_arrays} 中的第二个
宏就采用了这个思路。这里我们用它来得到前一个数组在对角线上的元素:
\begin{lstlisting}
> (with-array ((a 0 0) (d 1 1) (i 2 2)) ar
    (values a d i))
0
11
22
\end{lstlisting}

\begin{figure}
\begin{lstlisting}
(defmacro with-struct ((name . fields) struct &body body)
  (let ((gs (gensym)))
    `(let ((,gs ,struct))
       (let ,(mapcar #'(lambda (f)
                         `(,f (,(symb name f) ,gs)))
                     fields)
         ,@body))))
\end{lstlisting}
  \caption{结构体上的解构}
  \label{fig:destructuring_on_structures}
  \index{with-struct@\texttt{with-struct}}
\end{figure}

通过这个新宏,我们开始逐渐跳出那些认为元素必须以固定顺序出现的思维模式。我们可以做出一个类似形式的宏,
用它来绑定变量到~\texttt{defstruct} 所建立的结构体\index{destructuring!on structures}字段上。图
~\ref{fig:destructuring_on_structures} 中就这样定义一个宏。模式中的第一个参数被接受为与结构体相关联的
前缀,其余的都是字段名。为了建立访问调用,这个宏使用了~\texttt{symb}
(第~\pageref{fig:functions_which_operate_on_symbols_and_strings} 页)。
\begin{lstlisting}
> (defstruct visitor name title firm)
VISITOR
> (setq theo (make-visitor :name "Theodebert"
                           :title 'king
                           :firm 'franks))
#S(VISITOR NAME "Theodebert" TITLE KING FIRM FRANKS)
> (with-struct (visitor- name firm title) theo
    (list name firm title))
("Theodebert" FRANKS KING)
\end{lstlisting}
\index{Theodebert}

\section{引用}
\label{sec:destructuring:reference}
\index{destructuring!and reference}

\textsc{Clos} 自带了一个用于解构实例\index{destructuring!on instances}的宏。假设~\texttt{tree} 是一个带有三个~slot
的类:\texttt{species}、\texttt{age} 和~\texttt{height},而~\texttt{my-tree}
是一个~\texttt{tree} 的实例。在
\begin{lstlisting}
(with-slots (species age height) my-tree
  ...)
\end{lstlisting}
的里面我们可以像常规变量那样引用~\texttt{my-tree} 的这些~slot。在~\texttt{with-slots}\index{with-slots@\texttt{with-slots}}\index{Common Lisp!differences between versions!with-slots@\texttt{with-slots}}
的主体中,符号~\texttt{height} 指向~\texttt{height} slot。\texttt{height} 并不是简单地绑定到了对应~slot 里的变量,而是直接\emph{引用}到那个~slot 上。所以,如果我们写:
\begin{lstlisting}
(setq height 72)
\end{lstlisting}
那么也将给~\texttt{my-tree} 的~\texttt{height} 这个~slot 一个~72 的值。
这个宏的工作原理是
将~\texttt{height} 定义为一个展开到~slot 引用的符号宏~(第~\ref{sec:symbol_macros}
节)\index{symbol-macros}。
事实上,\texttt{symbol-macrolet} 就是为了支持像~\texttt{with-slots} 这样的宏才被
加入到~Common Lisp 中的。

无论~\texttt{with-slots} 事实上是不是一个解构宏,它在实际编程中所起的作用和~\texttt{destructuring-bind} 是一样的。
虽然通常的解构都是按值调用~(call-by-value),这种新型解构却是按名调用~(call-by-name)。
无论我们如何调用它,它对我们都是有用的。还有其他什么宏,我们可以依法炮制呢?

\begin{figure}
\begin{lstlisting}
(defmacro with-places (pat seq &body body)
  (let ((gseq (gensym)))
    `(let ((,gseq ,seq))
       ,(wplac-ex (destruc pat gseq #'atom) body))))

(defun wplac-ex (binds body)
  (if (null binds)
      `(progn ,@body)
      `(symbol-macrolet ,(mapcar #'(lambda (b)
                                     (if (consp (car b))
                                         (car b)
                                         b))
                                 binds)
         ,(wplac-ex (mapcan #'(lambda (b)
                                (if (consp (car b))
                                    (cdr b)))
                            binds)
                    body))))
\end{lstlisting}
  \caption{序列上的引用解构}
  \label{fig:reference_destructuring_on_sequences}
  \index{with-places@\texttt{with-places}}
\end{figure}

我们可以这样做:将解构宏展开成~\texttt{symbol-macrolet} 而不是~\texttt{let},这样,
就可以为任何解构宏创建出与之对应的按名调用版本。图~\ref{fig:reference_destructuring_on_sequences} 给出了
一个被修改成与~\texttt{with-slots} 行为类似的~\texttt{dbind} 版本。我们可以像使用
~\texttt{dbind} 一样来使用~\texttt{with-places}:
\begin{lstlisting}
> (with-places (a b c) #(1 2 3)
    (list a b c))
(1 2 3)
\end{lstlisting}
但这个新宏还给我们~\texttt{setf} 序列位置的选项,就像我们在~\texttt{with-slots} 里
所做的那样:
\begin{lstlisting}
> (let ((lst '(1 (2 3) 4)))
    (with-places (a (b . c) d) lst
      (setf a 'uno)
      (setf c '(tre)))
    lst)
(UNO (2 TRE) 4)
\end{lstlisting}
就像在~\texttt{with-slots} 里那样,这些变量现在都指向了序列中的对应位置。尽管如此,
这里还有一个重要的区别:你必须使用~\texttt{setf} 而不是~\texttt{setq} 来设置这些
伪变量。\texttt{with-slots} 宏必须引入一个~code-walker\index{code-walkers}
(第~\pageref{expl:code-walker}
页) 来将其体内的~\texttt{setq} 转化成~\texttt{setf}。这里,写一个~code-walker 将需要
写很多代码,但是带来的好处却不大。

如果~\texttt{with-places} 比~\texttt{dbind} 更通用,为什么不干脆只用它呢?
\texttt{dbind} 将一个变量关联一个值上,而~\texttt{with-places} 却是将变量关联到一组
用来找到一个值的指令集合上。每一个引用都需要进行一次查询。当~\texttt{dbind} 把
~\texttt{c} 绑定到~\texttt{(elt x 2)} 的值上时,\texttt{with-places} 将使
~\texttt{c} 成为一个展开成~\texttt{(elt x 2)} 的符号宏\index{symbol-macros}。所以如果~\texttt{c} 在宏体
中被求值了~$n$ 次,那将会产生~$n$ 次对~\texttt{elt} 的调用。除非你真的想要
~\texttt{setf} 那些由解构创建的变量,否则~\texttt{dbind} 将会更快一些。

\texttt{with-places} 的定义和~\texttt{dbind}
(图~\ref{fig:general_sequence_destructuring_operator}) 相比仅有轻微的变化。
在~\texttt{wplac-ex} (之前的~\texttt{dbind-ex}) 中那些~\texttt{let} 变成了
~\texttt{symbol-macrolet}。通过类似的改动,我们也可以为任何正常的解构宏做出一个按名调用
的版本。

\section{匹配}
\label{sec:destructuring:matching}

正如解构是赋值的泛化,\emph{模式匹配}是解构的泛化。“模式匹配”\index{pattern-matching}这个术语有许多含义。
在这里的语境中,它指的是这样的操作:比较两个结构,结构中可能含有变量,判断是否存在某种给变量赋值的
方式使得它们俩相等。例如,如果~\texttt{?x} 和~\texttt{?y} 是变量,那么这两个列表
\begin{lstlisting}
(p ?x ?y c ?x)
(p  a  b c  a)
\end{lstlisting}
当~\texttt{?x = a} 并且~\texttt{?y = b} 时匹配。而列表
\begin{lstlisting}
(p ?x b ?y a)
(p ?y b  c a)
\end{lstlisting}
当~\texttt{?x = ?y = c} 时匹配。

假设一个程序通过跟外部数据源交换信息的方式工作。为了回复一个消息,程序必须首先知道消息的类型
,并且还要取出它的特定内容。通过一个匹配操作符我们可以将这两步并成一步。

% xxx
要写出这种操作符,必须先想出一种区分变量的办法。我们不能直接把所有符号都当成
变量,因为需要让符号在模式中以参数的形式出现。这里我们规定:模式变量\index{pattern variables}是以问号开始的符号。
如果将来觉得不方便了,只要重定义谓词~\texttt{var?} 就可以改变这个约定。

\begin{figure}
\begin{lstlisting}
(defun match (x y &optional binds)
  (acond2
    ((or (eql x y) (eql x '_) (eql y '_)) (values binds t))
    ((binding x binds) (match it y binds))
    ((binding y binds) (match x it binds))
    ((varsym? x) (values (cons (cons x y) binds) t))
    ((varsym? y) (values (cons (cons y x) binds) t))
    ((and (consp x) (consp y) (match (car x) (car y) binds))
     (match (cdr x) (cdr y) it))
    (t (values nil nil))))

(defun varsym? (x)
  (and (symbolp x) (eq (char (symbol-name x) 0) #\?)))

(defun binding (x binds)
  (labels ((recbind (x binds)
             (aif (assoc x binds)
                  (or (recbind (cdr it) binds)
                      it))))
    (let ((b (recbind x binds)))
      (values (cdr b) b))))
\end{lstlisting}
  \caption{匹配函数}
  \label{fig:matching_function}
  \index{match@\texttt{match}}
  \index{varsym?@\texttt{varsym?}}
  \index{binding@\texttt{binding}}
\end{figure}

图~\ref{fig:matching_function} 包含一个模式匹配的函数,它跟一些~Lisp 入门读物里的匹配函数样子
差不多\note{238}。我们传给~\texttt{match} 两个列表,如果它们可以匹配,将得到另一个列表,该列表
会显示它们是如何匹配的:
\begin{lstlisting}
> (match '(p a b c a) '(p ?x ?y c ?x))
((?Y . B) (?X . A))
T
> (match '(p ?x b ?y a) '(p ?y b c a))
((?Y . C) (?X . ?Y))
T
> (match '(a b c) '(a a a))
NIL
NIL
\end{lstlisting}

% xxx
在~\texttt{match} 逐个元素地比较它的参数时,它建立起来了一系列值和变量
之间的赋值关系\index{binding list 绑定列表},这种关系被称为\emph{绑定}。这些变量是由参数~\texttt{binds} 给出的。
若匹配成功,\texttt{match} 返回其生成的绑定,
否则返回~\texttt{nil}。由于并非所有成功的匹配都能生成绑
定,所以和~\texttt{gethash} 一样,\texttt{match}用第二个返回值来表示
匹配成功与否:
\begin{lstlisting}
> (match '(p ?x) '(p ?x))
NIL
T
\end{lstlisting}
当~\texttt{match} 像上面那样返回~\texttt{nil} 和~\texttt{t} 时,它表示一个没有产生
任何绑定的成功的匹配。

和~Prolog 一样,\texttt{match} 也把~\texttt{\_} (下划线)\index{\_@\texttt{\_}} 用作通配符。它可以匹配任何
东西,并且对绑定没有任何影响:
\begin{lstlisting}
> (match '(a ?x b) '(_ 1 _))
((?X . 1))
T
\end{lstlisting}

\begin{figure}
\begin{lstlisting}
(defmacro if-match (pat seq then &optional else)
  `(aif2 (match ',pat ,seq)
         (let ,(mapcar #'(lambda (v)
                           `(,v (binding ',v it)))
                       (vars-in then #'atom))
           ,then)
         ,else))

(defun vars-in (expr &optional (atom? #'atom))
  (if (funcall atom? expr)
      (if (var? expr) (list expr))
      (union (vars-in (car expr) atom?)
             (vars-in (cdr expr) atom?))))

(defun var? (x)
  (and (symbolp x) (eq (char (symbol-name x) 0) #\?)))
\end{lstlisting}
  \caption{慢的匹配操作符}
  \label{fig:slow_matching_operator}
  \index{if-match@\texttt{if-match}}
  \index{var@\texttt{var?}}
\end{figure}

有了~\texttt{match},可以很容易地写出一个模式匹配版本的~\texttt{dbind}。
图~\ref{fig:slow_matching_operator} 中含有一个称为~\texttt{if-match} 的宏。
像~\texttt{dbind} 那样,它的前两个参数是一个模式和一个序列,然后它通过比较模式跟序列来
建立绑定。不过,它用另外两个参数取代了代码主体:一个~\texttt{then} 子句,在
新绑定下被求值,如果匹配成功的话;以及一个~\texttt{else} 子句在匹配失败时被求值。这里
有一个简单的使用~\texttt{if-match} 的函数:
\begin{lstlisting}
(defun abab (seq)
  (if-match (?x ?y ?x ?y) seq
    (values ?x ?y)
    nil))
\end{lstlisting}
如果匹配成功了,它将建立~\texttt{?x} 和~\texttt{?y} 的值,然后返回它们:
\begin{lstlisting}
> (abab '(hi ho hi ho)
HI
HO
\end{lstlisting}

函数~\texttt{vars-in} 返回一个表达式中的所有匹配变量。它调用~\texttt{var?} 来测试是否
某个东西是一个变量。目前,\texttt{var?} 和用来检测绑定列表中变量的~\texttt{varsym?}
(图~\ref{fig:matching_function}) 是相同的,之所以使用独立的两个函数是考虑到我们可能想要
给这两类变量采用不同的表示方法。

像在图~\ref{fig:slow_matching_operator} 里定义的那样,\texttt{if-match} 很短,
但并不是非常高效。它在运行期做的事情太多了。我们在运行期把两个序列都遍历了,尽管第一个
序列在编译期就是已知的。更糟糕的是,在进行匹配的过程中,我们构造列表来存放变量绑定。如果
充分利用编译期已知的信息,就能写出一个既不做任何不必要的比较,也不做任何~cons 的
~\texttt{if-match} 版本来。

如果其中一个序列在编译期已知,并且只有这个序列里含有变量,那么就要另做打算了。在一次
对~\texttt{match} 的调用中,两个参数都可能含有变量。通过将变量限制在~\texttt{if-match} 的
第一个参数上,就有可能在编译期知道哪些变量将会参与匹配。这里,我们不再创建变量绑定的列表,
而是将变量的值保存进这些变量本身。

\begin{figure}
\begin{lstlisting}
(defmacro if-match (pat seq then &optional else)
  `(let ,(mapcar #'(lambda (v) `(,v ',(gensym)))
                 (vars-in pat #'simple?))
     (pat-match ,pat ,seq ,then ,else)))

(defmacro pat-match (pat seq then else)
  (if (simple? pat)
      (match1 `((,pat ,seq)) then else)
      (with-gensyms (gseq gelse)
        `(labels ((,gelse () ,else))
           ,(gen-match (cons (list gseq seq)
                             (destruc pat gseq #'simple?))
                       then
                       `(,gelse))))))

(defun simple? (x) (or (atom x) (eq (car x) 'quote)))

(defun gen-match (refs then else)
  (if (null refs)
      then
      (let ((then (gen-match (cdr refs) then else)))
        (if (simple? (caar refs))
            (match1 refs then else)
            (gen-match (car refs) then else)))))
\end{lstlisting}
  \caption{快速匹配操作符}
  \label{fig:fast_matching_operator}
  \label{macro:if-match}
  \index{pat-match@\texttt{pat-match}}
\end{figure}

\begin{figure}
\begin{lstlisting}
(defun match1 (refs then else)
  (dbind ((pat expr) . rest) refs
    (cond ((gensym? pat)
           `(let ((,pat ,expr))
              (if (and (typep ,pat 'sequence)
                       ,(length-test pat rest))
                  ,then
                  ,else)))
          ((eq pat '_) then)
          ((var? pat)
           (let ((ge (gensym)))
             `(let ((,ge ,expr))
                (if (or (gensym? ,pat) (equal ,pat ,ge))
                    (let ((,pat ,ge)) ,then)
                         ,else))))
          (t `(if (equal ,pat ,expr) ,then ,else)))))

(defun gensym? (s)
  (and (symbolp s) (not (symbol-package s))))

(defun length-test (pat rest)
  (let ((fin (caadar (last rest))))
    (if (or (consp fin) (eq fin 'elt))
        `(= (length ,pat) ,(length rest))
        `(> (length ,pat) ,(- (length rest) 2)))))
\end{lstlisting}
  \caption{快速匹配操作符~(续)}
  \label{fig:fast_matching_operator_continued}
  \index{gensym?@\texttt{gensym?}}
  \index{typep@\texttt{typep}}
\end{figure}

在图~\ref{fig:fast_matching_operator} 和~\ref{fig:fast_matching_operator_continued} 中
是~\verb|if-match| 的新版本。如果能预见到哪部分代码会在运行期求值,
我们不妨就直接在编译期生成它。这里,我们生成的代码仅仅完成需要的那些比较操作,
而不是展开成对~\verb|match| 的调用。

如果我们打算使用变量~\texttt{?x} 来包含~\texttt{?x} 的绑定的话,怎样表达一个尚未被
匹配过程建立绑定的变量呢?这里,我们将通过将一个模式变量绑定到一个生成符号以表明其未绑定。
所以~\texttt{if-match} 一开始会生成代码将所有模式中的变量绑定到生成符号上。在这种情况下,
代替了展开成一个~\texttt{with-gensyms},在编译期做一次符号生成,然后将它们直接插入进展开式
是安全的。

其余的展开由~\texttt{pat-match} 完成。这个宏接受和~\texttt{if-match} 相同的参数;唯一
的区别是它不为模式变量建立任何新绑定。在某些情况下这是一个优点,
第~\ref{chap:a_query_compiler} 章将把~\texttt{pat-match} 作为一个独立的操作符来
使用。

在新的匹配操作符里,模式内容和模式结构之间的差别将用函数~\texttt{simple?} 定义。如果
我们想要在模式里使用字面引用,那么解构代码~(以及~\texttt{vars-in}) 必须被告知不要进入
那些第一个元素是~\texttt{quote} 的列表。在新的匹配操作符下,我们将可以使用列表作为
模式元素,只需简单地将它们引用起来。

与~\texttt{dbind} 相似,\texttt{pat-match} 调用~\texttt{destruc} 来得到一个
将要在运行期参与其参数调用的列表。这个列表被传给~\texttt{gen-match} 来为嵌套的模式递归
生成匹配代码,然后再传给~\texttt{match1},以生成模式树上每个叶子的匹配代码。

最后出现在一个~\texttt{if-match} 展开式中的多数代码都来自~\texttt{match1},如图
~\ref{fig:fast_matching_operator_continued}。这个函数分四种情况处理。
如果模式参数是一个生成符号,那么它是一个由~\texttt{destruc} 创建用于保存子列表的不可见
变量,并且所有我们需要在运行期做的就是测试它是否具有正确的长度。如果模式元素是一个通配符
~(\verb|_|),那么不需要生成任何代码。如果模式元素是一个变量,那么~\texttt{match1}
会生成代码去匹配,或者将其设置成,运行期给出的序列的对应部分。否则,模式元素被看作一个字面
上的值,而~\texttt{match1} 会生成代码去比较它和序列中的对应部分。

让我们通过例子来了解一下展开式中的某些部分的生成过程。假设我们从下面的表达式开始
\begin{lstlisting}
(if-match (?x 'a) seq
  (print ?x)
  nil)
\end{lstlisting}
这个模式将被传给~\texttt{destruc},同时带着一些生成符号~(不妨简称为~\texttt{g})
来代表那个序列:
\begin{lstlisting}
(destruc '(?x 'a) 'g #'simple?)
\end{lstlisting}
得到:
\begin{lstlisting}
((?x (elt g 0)) ((quote a) (elt g 1)))
\end{lstlisting}
在这个列表的开头我们接上~\texttt{(g seq)}:
\begin{lstlisting}
((g seq) (?x (elt g 0)) ((quote a) (elt g 1)))
\end{lstlisting}
然后把结果整个地发给~\texttt{gen-match}。就像~\texttt{length}
(第~\pageref{func:our-length_native} 页) 的原生实现那样,\texttt{gen-match} 首先
一路递归到列表的结尾,然后在回来的路上构造其返回值。当~\texttt{gen-match} 走完所有元素
时,它就返回其~\texttt{then} 参数,也就是~\texttt{(print ?x)}\footnote{译者注:
原文中说返回的~\texttt{then} 参数是~\texttt{?x},这应该是个笔误。}。
在递归回来的路上,这个返回值将作为~\texttt{then} 参数传给~\texttt{match1}。现在我们
将得到一个像这样的调用:
\begin{lstlisting}
(match1 '(((quote a) (elt g 1))) '(print ?x) '$\langle{}else function\rangle$)
\end{lstlisting}
得到:
\begin{lstlisting}
(if (equal (quote a) (elt g 1))
    (print ?x)
    $\langle{}else function\rangle$)
\end{lstlisting}
然后这些将成为另一个~\texttt{match1} 调用的~\texttt{then} 参数,得到的值将成为最后
的~\texttt{match1} 调用的~\texttt{then} 参数。这个~\texttt{if-match} 的完整展开式
显示在图~\ref{fig:expansion_of_an_if-match}
\footnote{译者注:原书里有一个笔误,展开式代码中的~\texttt{(gensym? x)}
  应为~\texttt{(gensym? ?x)}。} 中。

\begin{figure}
\begin{lstlisting}
(if-match (?x 'a) seq
  (print ?x))
\end{lstlisting}
展开成:
\begin{lstlisting}
(let ((?x '#:g1))
  (labels ((#:g3 nil nil))
    (let ((#:g2 seq))
      (if (and (typep #:g2 'sequence)
               (= (length #:g2) 2))
          (let ((#:g5 (elt #:g2 0)))
            (if (or (gensym? ?x) (equal ?x #:g5))
                (let ((?x #:g5))
                  (if (equal 'a (elt #:g2 1))
                      (print ?x)
                      (#:g3)))
                (#:g3)))
          (#:g3)))))
\end{lstlisting}
  \caption{一个~\texttt{if-match} 的展开式}
  \label{fig:expansion_of_an_if-match}
\end{figure}

在这个展开式里有两个地方用到了~gensym\index{gensym@\texttt{gensym}!as unbound} (生成符号),这两个地方的用意各不相同。在运行时,一些变量被用来保存树的一部分,
这些变量的名字是用~gensym 生成的,目的是为了避免捕捉。而变量~\texttt{?x} 在开始的时候被绑定到了
一个~gensym,以表明它尚未被匹配操作赋给一个值。\note{244-1}

在新的~\texttt{if-match} 中,模式元素现在是被求值而不再是被隐式引用了。\note{244-2}这意味着~Lisp
变量可以被用于模式中,和被引用的表达式一样:
\begin{lstlisting}
> (let ((n 3))
    (if-match (?x n 'n '(a b)) '(1 3 n (a b))
      ?x))
1
\end{lstlisting}
还有两个进一步的改进,是因为新版本调用了~\texttt{destruc}
(图~\ref{fig:general_sequence_destructuring_operator}) 而出现。现在模式中可以包含
~\texttt{\&rest} 或者~\texttt{\&body} 关键字~(\texttt{match} 是不管这一套的)。
并且因为~\texttt{destruc} 使用了一般的序列操作符\index{sequence operators}~\texttt{elt}\index{elt@\texttt{elt}} 和~\texttt{subseq}\index{subseq@\texttt{subseq}},
新的~\texttt{if-match} 将工作在任何类型的序列上。如果~\texttt{abab} 采用新版本来定义,
它也可以被用于向量\index{vectors 向量!matching}和字符串\index{strings 字符串!matching}:
\begin{lstlisting}
> (abab "abab")
#\a
#\b
> (abab #(1 2 1 2))
1
2
\end{lstlisting}
事实上,模式可以像~\texttt{dbind} 的模式那样复杂:
\begin{lstlisting}
> (if-match (?x (1 . ?y) . ?x) '((a b) #(1 2 3) a b)
    (values ?x ?y))
(A B)
#(2 3)
\end{lstlisting}
注意到,在第二个返回值里,向量的元素被显示出来了。要想使向量以这种方式被输出,需要将
~\texttt{*print-array*}\index{print-array@\texttt{*print-array*}} 设置为~\texttt{t}。

在本章,我们开始逐步走进一个崭新的编程领域。以一个简单的用于解构的宏作开端。
在~\texttt{if-match} 的最终版本中,我们有了某种看起来更像是它自己的语言的东西。
接下来的章节将要介绍一整类程序,它们秉承的都是相同的理念。

%%% Local Variables:
%%% coding: utf-8
%%% mode: latex
%%% TeX-master: "onlisp-cn"
%%% End:
