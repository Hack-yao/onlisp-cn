%!TEX encoding = UTF-8 Unicode
% $Id: 8-when_to_use_macros.tex 18 2014-03-12 22:35:24Z binghe $

\chapter{何时使用宏}
\label{chap:when_to_use_macros}
\index{macros 宏!when to use}

我们如何知道一个给定的函数是否真的应该是函数,而不是宏呢?多数时候,会很容易分清楚
在哪种情况下需要用到宏,哪种情况不需要。缺省情况下,我们应
该用函数,因为如果函数能解决问题,而偏要用上宏的话,会让程序变得不优雅。
我们应当只有在宏能带来特别的好处时才使用它们。

什么情况下,宏能给我们带来优势呢?这就是本章的主题。通常这不是锦上添花,而是一种必须。
大多数我们用宏可以做到的事情\index{macros 宏!unique powers of},函数都无法完成。第
~\ref{sec:when_nothing_else_will_do} 节列出了只能用宏来实现的几种操作符。\note{106}尽管如
此,也有一小类~(但很有意思的) 情况介于两者之间,对它们来说,不管把操作符实现成函数还是宏似乎都言之有理。对于这种情况,第~\ref{sec:macro_or_function} 节给出了关于宏的正反
两方面考量。最后,在充分考察了宏的能力后,我们在第
~\ref{sec:applications_for_macros} 节里转向一个相关问题:人们都用宏干什么?

\section{当别无他法时}
\label{sec:when_nothing_else_will_do}

优秀设计的一个通用原则就是:当你发现在程序中的几处都出现了相似的代码时,就应该写
一个子例程,并把那些相似的语句换成对这个子例程的调用。如果也把这条原则
用到~Lisp 程序上,就必须先决定这个``子例程''应该是函数还是宏。

有时,可以很容易确定应当写一个宏而不是函数,因为只有宏才能满足需求。一个像
~\texttt{1+} 这样的函数或许既可以写成函数也可以写成宏:
\begin{lstlisting}
(defun 1+ (x) (+ 1 x))

(defmacro 1+ (x) `(+ 1 ,x))
\end{lstlisting}
但是来自第~\ref{sec:defining_simple_macros} 节的~\texttt{while},则只能被定义成
宏:
\begin{lstlisting}
(defmacro while (test &body body)
  `(do ()
       ((not ,test))
     ,@body))
\end{lstlisting}
无法用函数来重现这个宏的行为。\texttt{while} 的定义里拼接了一个作为
~\texttt{body} 传入~\texttt{do} 的主体里的表达式,它只有当~\texttt{test} 表达式
返回~\texttt{nil} 时才会被求值。没有函数可以做到这一点;是因为在函数调用里,所有的
参数在函数调用开始之前就会被求值。

当你需要用宏时,你看中了它哪一点呢?宏有两点是函数无法做到的:宏可以
控制~(或阻止) 对其参数\index{macros 宏!arguments to 的参数}的求值,并且它可以展开进入到主调方的上下文中。任何
需要宏的应用,归根到底都是要用上述两个属性中的至少一个。

``宏不对其参数进行求值'',这个非正式的说法不太准确。更确切的说法应该是,``宏能
\emph{控制}宏调用中参数的求值''。取决于参数在宏展开式中的位置,它们可以被
求值一次,多次,或者根本不求值。宏的这种控制主要体现在四个方面:

\begin{enumerate}

\item \emph{变换。} \index{transformation!of macro
  arguments} Common Lisp 的~\texttt{setf} 宏就是这类宏中的一员,它们在
  求值前都会对传入的参数严加检查。内置的访问函数~(access function) 通常
  都有一个对应的逆操作,其作用是对该访问函数所获取的对象赋值。
  \texttt{car} 的逆操作是~\texttt{rplaca},对于~\texttt{cdr} 来说
  是~\texttt{rplacd},等等。有了~\verb|setf|,我们就可以把对这些访问函数的调用
  当成变量赋值。\texttt{(setf (car x) 'a)} 就是个例子,这个表达式可以展开
  成~\verb|(progn (rplaca x 'a) 'a)|.

  为了有这样的效果,\verb|setf| 必须非常了解它的第一个参数。如果要知
道上述的情况需要用到~\verb|rplaca|,\verb|setf| 就得清楚它的第一
个参数是个以~\verb|car| 开始的表达式。这样的话,\verb|setf| 以及其
他修改参数的操作符,就必须被写成宏。

\item \emph{绑定。}词法变量必须在源代码中直接出现。例如,由于
  ~\verb|setq| 的第一个参数是不求值的,所以,所有在~\verb|setq| 之上构建的东西
  都必须是展开到~\texttt{setq} 的宏,而不能是调用它的函数。对于
  ~\texttt{let} 这样的操作符也是如此,它的实参必须作为~\texttt{lambda} 表达式的
  形参出现,还有类似~\texttt{do} 这样展开到~\texttt{let} 的宏也是这样,等等。任何新
  操作符,只要它修改了参数的词法绑定\index{bindings, altering},那么它就必须写成宏。

\item \emph{条件求值。}函数的所有参数都会被求值。在
  像~\verb|when| 这样的结构里,我们希望一些参数仅在特定条件下才被求值。只有通过
  宏才可能获得这种灵活性\index{iteration!macros for}。

\item \emph{多重求值。}函数的所有参数不但都会被求值,而且
  求值的次数都正好是一次。我们需要用宏来定义像~\verb|do| 这样的结构,这样子,就可以
  对特定的参数多次求值。
\end{enumerate}
也有几种方式可以利用宏产生的内联展开式带来的优势。这里必须强调一点,
宏展开后生成的展开式将会出现在宏调用所在的词法环境\index{environment!of macro expansions}\index{macros 宏!environment of expansion}之中,因为下列三种
用法有两种都基于这个事实。它们是:
\begin{enumerate}
\setcounter{enumi}{4}
\item\label{itm:using-calling-env} \emph{利用调用方环境。}宏生成的展开式可以含有这样的变量,变量的绑定来自宏调用的上下文环境。下面这个宏:
\begin{lstlisting}
(defmacro foo (x)
  `(+ ,x y))
\end{lstlisting}
的行为将因~\verb|foo| 被调用时~\verb|y| 的绑定而不同。

这种词法交流\index{intercourse, lexical 词法交流}通常更多地被视为瘟疫的传染源,
而非快乐之源。一般来说,写这样的宏不是什么好习惯。
函数式编程的思想对于宏也同样适用:与一个宏交流的最佳方式就是通过它的参数。
事实上,需要用到调用方环境的情况极少,因此,如果出现了这样的用法,那十有八九就是
什么地方出了问题。(见第~\ref{chap:variable_capture} 章) 纵观本书中的所有宏,只有
续延传递~(continuation--passing) 宏~(第~\ref{chap:continuations} 章) 和
~\textsc{atn} 编译器~(第~\ref{chap:parsing_with_atns} 章) 的一部分以这种方式
利用了调用方环境。

\item\label{itm:wrapping-new-env} \emph{包装新环境。}宏也可以使其参数在
一个新的词法环境下被求值。最经典的例子就是~\verb|let|,它可以用~\verb|lambda|
实现成宏的形式~(见~\pageref{fig:macro_implementation_of_let} 页)。
在一个~\verb|(let ((y 2)) (+ x y))| 这样的表达式里,\verb|y| 将指向一个新的变量。

\item \emph{减少函数调用。}宏展开后,展开式内联地插入展开环境。这个设计的第三个结果是宏调用在编译后的代码中没有额外开销。\index{function calls, avoiding! with macros}到了运行期,宏调用已经替换成了它的展开式。(这个说法对于声明成~\verb|inline| 的函数也一样成立。)
\end{enumerate}
很明显,如果不是有意为之,情形~\ref{itm:using-calling-env} 和~\ref{itm:wrapping-new-env} 将产生变量捕捉上的问题,这可能是宏的编写者所有担心的事情里面最头疼的一件。变量捕捉将在第
~\ref{chap:variable_capture} 章讨论。

与其说有七种使用宏的方式,不如说有六个半。在理想的世界\index{world 世界, ideal 理想的}里,所有~Common Lisp
编译器都会遵守~\verb|inline| 声明,所以减少函数调用将是内联函数的职责,而不是
宏的。这个建立理想世界的重任就作为练习留给读者吧。

\section{宏还是函数?}
\label{sec:macro_or_function}
\index{functions 函数!vs. macros 宏}
\index{macros 宏!vs. functions 函数}

上一节解决了较简单的一类问题。一个操作符,倘若在参数被求值前就需要访问它,那么这个操作符就应该写成宏,因为别无他法。那么,如果有操作符用两种写法都能实现,那该怎么办呢?比如说操作符~\verb|avg|,它返回参数的平均值。它可以定义成函数
\begin{lstlisting}
(defun avg (&rest args)
  (/ (apply #'+ args) (length args)))
\end{lstlisting}
但把它定义成宏也不错:
\begin{lstlisting}
(defmacro avg (&rest args)
  `(/ (+ ,@args) ,(length args)))
\end{lstlisting}
因为每次调用~\verb|avg| 的函数版本时,都毫无必要地调用了一次~\verb|length|。在编译期我们可能不清楚这些参数的值,但却知道参数的个数,所以那是调用~\verb|length| 最佳的时机。当我们面临这样的选择时,可以考虑下列几点:

\begin{center}\large\textsl{利}\end{center}

\begin{enumerate}
\item \emph{编译期计算。}
  宏调用共有两次参与计算,分别是:宏展开的时候,以及展开式被求值的时候。一旦程序编译好,Lisp~程序中所有的宏展开也就完成了,而在编译期每进行一次计算,都帮助程序在运行的时候卸掉了一个包袱。如果在编写操作符时,可以让它在宏展开的阶段就完成一部分工作,那么把
  它写成宏将会让程序更加高效。因为只要是聪明的编译器无法自己完成的工作,
  函数就只能把这些事情拖到运行期做。第~\ref{chap:computation_at_compile-time} 章
  介绍一些类似~\texttt{avg} 的宏,这些宏能在宏展开的阶段就完成一部分工作。
\item \emph{和~Lisp 的集成。}\index{Lisp!integration with user programs}
  有时,用宏代替函数可以令程序和~Lisp 集成得更紧密。解决一个特定问题的方法,可以
  是专门写一个程序,你也可以用宏把这个问题变换成另一个~Lisp 已经知道解决办法的问题。
  如果可行的话,这种方法常常可以使程序变得更短小,也更高效:更小是因为~Lisp 代劳
  了一部分工作,更高效则是因为产品级~Lisp 系统通常比用户程序做了更多的优化。这一优势
  大多时候会出现在嵌入式语言里,而我们从第~\ref{chap:a_query_compiler} 章起会全面转向
  嵌入式语言。
\item \emph{免除函数调用。}
  宏调用在它出现的地方直接展开成代码。所以,如果你把常用的代码片段写成宏,
  那么就可以每次在使用它的时候免去一次函数调用。\index{function calls, avoiding!by inline compilation}在~Lisp 的早期方言中,程序员
  借助宏的这个属性在运行期避免函数调用。而在~Common Lisp 里,这个差事应该由声明成
  ~\texttt{inline} 类型的函数接手了。

  通过将函数声明成~\texttt{inline},你要求把这个函数就像宏一样,直接编译进调用方的代码。
  不过,理想和现实还是有距离的;\textsc{cltl}2 (229 页)~说
  ~``编译器可以随意地忽略该声明'',而且某些~Common Lisp 编译器确实也是这样做的。
\end{enumerate}

在某些情况下,效率因素和跟~Lisp 之间紧密集成的组合优势可以充分证实使用宏的必要性。
在第~\ref{chap:a_query_compiler} 章的查询编译器里,可以转移到编译期的计算量
相当可观,这使我们有理由把整个程序变成一个独立的巨型宏。尽管效率是初衷,这一
转移同时也让程序和~Lisp 走得更近:在新版本里,能更容易地使用~Lisp 表达式,比如说
可以在查询的时候用~Lisp 的算术表达式。

\begin{center}\large\textsl{弊}\end{center}
\begin{enumerate}
\setcounter{enumi}{3}
\item \emph{函数即数据}, 而宏在编译器看来,更像是一些
  指令。函数可以当成参数传递~(例如用~\texttt{apply}\index{macros 宏!and apply@and \texttt{apply}}),被函数返回,或者保存在
  数据结构里。但这些宏都做不到。\label{the_cons_4}

  有的情况下,你可以通过将宏调用封装在~lambda--表达式里来达到目的。
  如果你想用~\texttt{apply}\index{apply@\texttt{apply}!with macros 和宏一起用} 或~\texttt{funcall} 来调用某些
  的宏,这样是可行的,例如:
\begin{lstlisting}
> (funcall #'(lambda (x y) (avg x y)) 1 3)
2
\end{lstlisting}
不过这样做还是有些麻烦。而且它有时还无法正常工作:如果这个宏
带有~\verb|&rest| 形参,那么就无法给它传递可变数量的实参,\texttt{avg} 就是个例子。
\item \emph{源代码清晰。}宏定义和等价的函数定义相比
  更难阅读。所以如果将某个功能写成宏只能稍微改善程序,那么最好还是改成使用
  函数。
\item\label{itm:clarity-at-runtime} \emph{运行期清晰。}宏有时比函数更难调
  试。如果你在含有许多宏的代码里碰到运行期错误,那么你
  在~backtrace\index{backtraces} 里看到的代码将包含
  所有这些宏调用的展开式,而它们和你最初写的代码看起来可能会大相径庭。\par
  并且由于宏展开以后就消失了,所以它们在运行时是看不到的。你不是总能使
  用~\texttt{trace}\index{trace@\texttt{trace}} 来分析一个宏的调用过程。
  如果~\texttt{trace} 真的奏效的话,它展示给你的只是对宏展开函数的调
  用,而非宏调用本身的调用。
 
\item \emph{递归。}在宏里使用递归不像在函数里那么简单。尽管展开
  一个宏里的展开函数可能是递归的,但展开式本身可能不是。第~\ref{sec:recursion}
  节将处理跟宏里的递归有关的主题。
\end{enumerate}

在决定何时使用宏的时候需要权衡利弊,综合考虑所有这些因素。只有靠经验才能知道哪一个因素在起主导作用。
尽管如此,出现在后续章节里的宏的示例涵盖了大多数对宏有利的情形。如果一个潜在的
宏符合这里给出的条件,那么把它写成这样可能就是合适的。

最后,应该注意运行期清晰~(观点~\ref{itm:clarity-at-runtime}) 很少成为障碍。调试那种用很多宏写成的代码并不像你
想象的那样困难。如果一个宏的定义长达数百行,在运行期调试它的展开式的确是件苦差事。
但至少\utility{}往往出现在小而可靠的程序层次中。通常它们的定义长度不超过~15 行。
所以就算你最终只得仔细检查一系列的~backtrace,这种宏也不会让你云遮雾绕,摸不着头脑。

\section{宏的应用场合}
\label{sec:applications_for_macros}
\index{macros 宏!applications of}

在了解了宏的十八般武艺之后,下一个问题是:我们可以把宏用在哪一类程序里?
关于宏的用途,最正式的表述可能是:它们主要用于句法转换~(syntactic
transformations)。这并不是要严格限制宏的使用范围。由于~Lisp 程序从
列表中生成\footnote{\emph{从列表中生成},是指列表作为编译器的输入。函数不再
\emph{从列表中生成},虽然在一些早期的方言里的确是这样处理的。},而列表是
~Lisp 数据结构,``句法转换'' 的确有很大的发挥空间。
第~\ref{chap:a_query_compiler}--\ref{chap:prolog} 章展示的整个程序,其目的就可以
说成``句法转换'',而且从效果上看,所有宏莫不是如此。

% xxx
宏的种种应用一起织成了一条缎带,这些应用涵盖了从像~\texttt{while} 这样小型通用的宏,直到后面
章节定义的大型、特殊用途的宏。缎带的一端是\emph{\utility{}},它们和每个~Lisp 都内置的
那些宏是一样的。它们通常短小、通用,而且相互独立。尽管如此,你也可以为一些特别类型的
程序编写\utility{},然后当你有一组宏用于,比如说,图形程序的时候,它们看起来就像
是一种专门用于图形编程的语言。在缎带的远端,宏允许你用一种和~Lisp 截然不同的语言
来编写整个程序。以这种方式使用宏的做法被称为实现\emph{嵌入式语言}。

\utility{}是自底向上风格的首批成果。甚至当一个程序规模很小而不必分层构建时,它也
仍然能够对程序的最底层,即~Lisp 本身加以扩充,并从中获益。\verb|nil!| 将其参数设置为
~\texttt{nil},这个\utility{}只能定义成宏:
\begin{lstlisting}
(defmacro nil! (x)
  `(setf ,x nil))
\end{lstlisting}
看到~\texttt{nil!},可能有人会说它什么都\emph{做}不了,无非可以让我们少输入\index{typing 输入}几个字
罢了。是的,但是充其量,宏所能做的也就是让你少打些字而已。如果有人非要这样想的话,那么
其实编译器的工作也不过是让人们用机器语言编程的时候可以少些。不可低估\utility{}的价值,
因为它们的功用会积少成多:几层简单的宏拉开了一个优雅的程序和一个晦涩的程序之间的差距。

多数\utility{}都含有模式。当你注意到代码中存在模式时,不妨考虑把它写成
\utility{}。模式是计算机最擅长的。为什么有程序可以代劳,还要自己动手呢?
假设在写某个程序的时候,你发现自己以同样的通用形式在很多地方做循环操作:
\begin{lstlisting}
(do ()
    ((not $\langle$condition$\rangle$))
  . $\langle$body of code$\rangle$)
\end{lstlisting}
当你在自己的代码里发现一个重复的模式时,这个模式经常会有一个名字。这里,模式的名字
是~\emph{while}。如果我们想把它作为\utility{}提供出来,那么只能以宏的形式,因为
需要用到带条件判断的求值,和重复求值。倘若用第~\pageref{macro:while} 页的定义
实现~\texttt{while},如下:
\begin{lstlisting}
(defmacro while (test &body body)
  `(do ()
       ((not ,test))
     ,@body))
\end{lstlisting}
就可以将该模式的所有实例替换成:
\begin{lstlisting}
(while $\langle$condition$\rangle$
  . $\langle$body of code$\rangle$)
\end{lstlisting}
这样做使得代码更简短,同时也更清晰地表明了程序的意图。

宏的这种变换参数的能力使得它在编写接口时特别有用。适当的宏可以在本应需要输入冗长
复杂表达式的地方只输入简短的表达式。尽管图形界面减少了为最终用户编写这类宏的
需要,程序员却一直使用这种类型的宏。最普通的例子是~\texttt{defun}\index{defun@\texttt{defun}},在表面上,
它创建的函数绑定类似用~Pascal 或~C 这样的语言定义的函数。第
~\ref{chap:functions} 章提到下面两个表达式差不多具有相同的效果:
\begin{lstlisting}
(defun foo (x) (* x 2))

(setf (symbol-function 'foo)
      #'(lambda (x) (* x 2)))
\end{lstlisting}
这样~\texttt{defun} 就可以实现成一个将前者转换成后者的宏。我们可以想象它会这样
写:
\begin{lstlisting}
(defmacro our-defun (name parms &body body)
  `(progn
     (setf (symbol-function ',name)
           #'(lambda ,parms (block ,name ,@body)))
     ',name))
\end{lstlisting}

像~\texttt{while} 和~\texttt{nil!} 这样的宏可以被视为通用的\utility{}。任何~Lisp
程序都可以使用它们。但是特定的领域同样也可以有它们自己的\utility{}。没有理由认为扩展
编程语言的唯一平台只能是原始的~Lisp。举个例子,如果你正在编写一个~\textsc{cad} 程序,
有时,最佳的实现可能会把它写成两层:一门专用于~\textsc{cad} 程序的语言
~(或者如果你偏爱更现代的说法,一个工具箱~(toolkit)),以及在这层之上的,你的特定应用。

Lisp 模糊了许多对其他语言来说理所当然的差异。在其他语言里,在编译期和运行期,
程序和数据,以及语言和程序之间具有根本意义上的差异。而在~Lisp 里,这些差异就
退化成了口头约定。例如,在语言和程序之间就没有明确的界限。你可以根据手头程序
的情况自行界定。因而,是把底层代码称作工具箱,还是称之为语言,确实不过是个说法而已。
将其视为语言的一个好处是,它暗示着你可以扩展这门语言,
就像你通过\utility{}\index{utilities 实用函数!become languages 成为语言}来扩展~Lisp 一样。

\begin{figure}
\begin{lstlisting}
(defun move-objs (objs dx dy)
  (multiple-value-bind (x0 y0 x1 y1) (bounds objs)
    (dolist (o objs)
      (incf (obj-x o) dx)
      (incf (obj-y o) dy))
    (multiple-value-bind (xa ya xb yb) (bounds objs)
      (redraw (min x0 xa) (min y0 ya)
              (max x1 xb) (max y1 yb)))))

(defun scale-objs (objs factor)
  (multiple-value-bind (x0 y0 x1 y1) (bounds objs)
    (dolist (o objs)
      (setf (obj-dx o) (* (obj-dx o) factor)
            (obj-dy o) (* (obj-dy o) factor)))
    (multiple-value-bind (xa ya xb yb) (bounds objs)
      (redraw (min x0 xa) (min y0 ya)
              (max x1 xb) (max y1 yb)))))
\end{lstlisting}
\caption{\label{fig:original_move_and_scale}最初的平移和缩放}
\end{figure}

设想我们正在编写一个交互式的~2D 绘图程序。为了简单起见,我们将假定程序处理的对象
只有线段,每条线段都表示成一个起点~$\langle x,y \rangle$ 和一个向量~$\langle dx,dy \rangle$。
并且我们的绘图程序的任务之一
是平移一组对象。这正是图~\ref{fig:original_move_and_scale} 中函数
~\texttt{move-objs} 的任务。出于效率考虑,我们不想在每个操作结束后重绘整个屏幕
\pozhehao{}只画那些改变了的部分。因此两次调用了函数~\texttt{bounds},它返回表示
一组对象的矩形边界的四个坐标~(最小~x,最小~y,最大~x,最大~y)。
\texttt{move-objs} 的操作部分被夹在了两次对~\texttt{bounds} 调用的中间,它们
分别找到平移前后的矩形边界,然后重绘整个区域。

函数~\texttt{scale-objs} 被用来改变一组对象的大小。由于区域边界可能随缩放因子的
不同而放大或者缩小,这个函数也必须在两次~\texttt{bounds} 调用之间发生作用。随着我们
绘图程序开发进度的不断推进,这个模式一次又一次地出现在我们眼前:在旋转,翻转,转置等函数里。

\begin{figure}
\begin{lstlisting}
(defmacro with-redraw ((var objs) &body body)
  (let ((gob (gensym))
        (x0 (gensym)) (y0 (gensym))
        (x1 (gensym)) (y1 (gensym)))
    `(let ((,gob ,objs))
       (multiple-value-bind (,x0 ,y0 ,x1 ,y1) (bounds ,gob)
         (dolist (,var ,gob) ,@body)
         (multiple-value-bind (xa ya xb yb) (bounds ,gob)
           (redraw (min ,x0 xa) (min ,y0 ya)
                   (max ,x1 xb) (max ,y1 yb)))))))

(defun move-objs (objs dx dy)
  (with-redraw (o objs)
    (incf (obj-x o) dx)
    (incf (obj-y o) dy)))

(defun scale-objs (objs factor)
  (with-redraw (o objs)
    (setf (obj-dx o) (* (obj-dx o) factor)
          (obj-dy o) (* (obj-dy o) factor))))
\end{lstlisting}
\caption{\label{fig:move_and_scale_filleted}骨肉分离后的平移和缩放}
\index{functions 函数!filleting 骨肉分离}
\end{figure}

通过一个宏,我们可以把这些函数中相同的代码抽象出来。
图~\ref{fig:move_and_scale_filleted} 中的宏~\texttt{with-redraw} 给出了一个框架,
它是图~\ref{fig:original_move_and_scale} 中几个函数所共有的。\footnote{这个宏的定义
使用了下一章才出现的~\texttt{gensym}。它的作用接下来就会说明。} 这样的话,这些函数每一个
的定义都缩减到了四行代码,如图~\ref{fig:move_and_scale_filleted}
末尾所示。通过这两个函数,这个新写的宏在简洁性方面作出的贡献证明了它是物有所值的。
\note{115}并且,一旦把屏幕重绘的细节部分抽象出来,这两个函数就变得清爽多了。

对~\texttt{with-redraw}\label{mac:with-redraw},有一种看法是把它视为一种语言的控制结构,
这种语言专门用于编写交互式的绘图程序。随着我们开发出更多这样的宏,它们不管从名义上,
还是在实际上都会构成一门专用的编程语言,并且我们的程序也将开始表现出其不俗之处,
这正是我们用特制的语言撰写程序所期望的效果。

宏的另一主要用途就是实现嵌入式语言。Lisp 在编写编程语言方面是一种特别优秀的
语言,因为~Lisp 程序可以表达成列表\index{lists!as code},而且~Lisp 还有内置的解析器~(\texttt{read})
和编译器~(\texttt{compile}\index{compile@\texttt{compile}}) 可以用在以这种方式表达的程序中。多数时候甚至不用调用
~\texttt{compile};你可以通过编译那些用来做转换的代码~(第~\pageref{page:compile} 页)\index{transformation!embedded languages implemented by},让你的嵌入式语言在无形中完成编译。

与其说嵌入式语言是构建于~Lisp 之上的语言,不如说它是和~Lisp 融为一体的,
这使得其语法成为了一个~Lisp 和新语言中特有结构的混合体。
实现嵌入式语言的初级方式是用~Lisp 给它写一个解释器。
有可能的话,一个更好的方法是通过语法转换实现这种语言:将每个表达式转换成
~Lisp 代码,然后让解释器可以通过求值的方式来运行它。这就是宏大展身手的时候了。宏的工作恰
恰是将一种类型的表达式转换成另一种类型,所以在编写嵌入式语言时,宏是最佳人选。

一般而言,嵌入式语言可以通过转换实现的部分越多越好。主要原因是可以节省工作量。
举个例子,如果新语言里含有数值计算,那你就无需面对表示和处理数值量的所有细枝末节。
如果~Lisp 的计算功能可以满足你的需要,那么你可以简单地将你的算术表达式转换成
等价的~Lisp 表达式,然后将其余的留给~Lisp 处理。

代码转换通常都会提高你的嵌入式语言的效率。而解释器在速度方面却一直处于劣势。当代码里
出现循环时,通常每次迭代解释器都必须重新解释代码,而编译器却只需做一次编译。
因此,就算解释器本身是编译的,使用解释器的嵌入式语言也会很慢。但如果新语言里
的表达式被转换成了~Lisp,那么~Lisp 编译器就会编译这些转换出来的代码。
这样实现的语言不需要在运行期承受解释的开销。要是你还没有为你的语言编写
一个真正编译器,宏会帮助你获得最优的性能。事实上,转换新语言的宏可以看作该语言的编译器
\pozhehao{}只不过它的大部分工作是由已有的~Lisp 编译器完成的。

这里我们暂时不会考虑任何嵌入式语言的例子,第~\ref{chap:a_query_compiler}--%
\ref{chap:object-oriented_lisp} 章都是关于该主题的。
第~\ref{chap:a_query_compiler} 章专门讲述了解释与转换嵌入式语言之间的区别,
并且同时用这两种方法实现了同一种语言。

有一本~Common Lisp 的书断言宏的作用域是有限的,依据是:
在所有~\textsc{cltl}1 里定义的操作符中,只有少于~$10\%$ 的操作符是宏。这就好比
是说因为我们的房子是用砖砌成的,我们的家具也必须得是\index{bricks, furniture made of 用砖头砌的家具}。宏在一个~Common Lisp 程序
中所占的比例多少\index{macros 宏!proportion in a program}完全要看这个程序想干什么。有的程序里可能根本没有宏,而有的程序可能全是宏。

%%% Local Variables:
%%% coding: utf-8
%%% mode: latex
%%% TeX-master: "onlisp-cn"
%%% End:
